%%%%%%%%
%%
%% Tex code representing Design Principles Table
%%
\def\arraystretch{0} 
\begin{xltabular}[l]{\linewidth}{@{} >{\small\raggedright\arraybackslash}p{0.47\linewidth} >{\small\raggedright\arraybackslash}X @{}}

\caption{The sIOP design principles; structured according to \cite{Greefhorst2011} \label{tab:dps}} \\
\toprule
\multicolumn{1}{@{}l}{Design Principle} & \multicolumn{1}{l}{Dimensions \& Attributes} \\ 
\multicolumn{1}{l}{Rationale} & \multicolumn{1}{l}{\quad Implications} \\ \cmidrule(r){1-1} \cmidrule(l){2-2}
\endfirsthead
\multicolumn{2}{@{}l}{\ldots\ \tiny Continuation}\\ \midrule
\multicolumn{1}{@{}l}{Design Principle} & \multicolumn{1}{l}{Dimensions \& Attributes} \\ 
\multicolumn{1}{l}{Rationale} & \multicolumn{1}{l}{\quad Implications} \\ \cmidrule(r){1-1} \cmidrule(l){2-2}
\endhead
\midrule\multicolumn{2}{r@{}}{\tiny Continued \ldots}\\
\endfoot
\endlastfoot
%
%
%
\begin{mmdp}\label{dp:shf}{\bfseries Semantic heterogeneity principle:}
\quad sIOP should strive to actively reconcile semantic differences as opposed to enforce semantic homgeneity.
\end{mmdp}
&
\begin{description}[labelwidth=3.7cm,leftmargin=3.7cm+1ex,nosep,topsep=2ex,labelsep=1ex,font=\bfseries]
  \item[Type of information:] business, data  
  \item[Quality attributes:] semantics, interoperability, reliability, reusability, efficiency
\end{description} \\
\begin{enumerate}[left=6pt, nosep]
  \item The user of the software agent has a natural, business-controlled perspective on the DoA;
  \item Data represent the state of affairs about that DoA, viewed from a particular perspective of use;
  \item Semantics of data, and with it the faithfulness with which the data refers to reality, is directly dependent on the perspective of use;
  \item Such view is just one particular perspective out of many equally legitimate ones;
  \item Equally legitimate perspectives on reality naturally create semantic heterogeneity;
  \item Accepting semantic heterogeneity implies the probable uniqueness of the agents view on reality;
  \item Without adherence to this principle, sIOP can be achieved only for dedicated DSC/DSP pairs, with collaborations that have been foreseen, for which the semantics are assumed stable over time, implemented with technology or platforms for which no or limited evolution is anticipated, and assuming that new communication channels won't emerge over time.
\end{enumerate}
&
\begin{enumerate}[left=10pt, nosep]
  \item Semantic heterogeneity maximises semantic clarity, accuracy and, consequently, the faithfulness of the representation of the SoA in reality;
  \item Semantic heterogeneity allows for a development of the software agent independently from other, potentially collaborative software agents;
  \item A software agent needs to specify the semantics of its data elements only once whilst remaining prepared for any number of distinct communication peers;
  \item Semantic heterogeneity weakens the need to coordinate semantic specifications centrally; in stead, “anyone can say anything about any topic”, resulting in formalising the current state that semantic definitions are highly distributed;
  \item Maintenance and evolution of its semantics remain under the DSP’s own control throughout the whole lifespan of the agent, and only depends on its business interest for investment, no matter the growth and evolution of the domain;
  \item Semantic heterogeneity demands availability of semantic scalability solutions;
  \item Accepting semantic heterogeneity implies acceptance of semantic mediation and resulting semantic alignments in order to achieve sIOP.
\end{enumerate} \\
%
%
%
\begin{mmdp}\label{dp:rfsm}{\bfseries Semantic responsibility principle:}
\quad When it is reasonable to expect that the software agent will be engaged in collaboration or otherwise will interoperate with (an)other software agent(s), it is the responsibility of the DSP to serve the quantity and manner of the potential interoperability by specifying its DM, and particularly the semantic meaning of its data, in advance.
\end{mmdp}
&
\begin{description}[labelwidth=3.7cm,leftmargin=3.7cm+1ex,nosep,topsep=2ex,labelsep=1ex,font=\bfseries]
  \item[Type of information:] Business, Data;
  \item[Quality attributes:] Semantics, interoperability, usability, efficiency;
\end{description} \\
\begin{enumerate}[left=6pt, nosep]
  \item Computers are not capable of genuine understanding, hence cannot establish semantics from data and thus require the human-in-the-loop for that;
  \item The user of the software agent has a natural, business controlled perspective on the DoA;
  \item This particular perspective is mirrored by the design of the software agent in the way how the entities of interest are carved out from the DoA. This includes the background knowledge that applies, the used terminology and how data from these entities are to be processed;
  \item Consequently, the responsibility for formulating the semantics that are expressed by the data can only lay by the DSP;
  \item On specifying semantics, Grice’s maxims on communication, and particularly on serving the quantity and manner of communication, represent the natural constraints to respect;
  \item Without adherence to this principle, the meaning of the data expressed by the software agent can be considered flawed, inaccurate, incomplete or otherwise insufficient in its support for semantic interoperability.
\end{enumerate}
&
\begin{enumerate}[left=10pt, nosep]
  \item The specification of the data semantics is only dependent on the agent’s own perspective on the application domain, and can therefore be fulfilled without any specific demands on interoperability with other software agents;
  \item No matter the number of different communication peers, the software agent needs to have the semantics of its data elements specified only once;
  \item By providing an explicit semantic specification of the data in advance, a software agent is prepared to engage in sIOP with other agents that can now connect to its semantics;
  \item By establishing collaboration with other DSPs or DSCs through its DM, the software agent ensures that the communication is grounded in its own perspective.
\end{enumerate} \\
%
%
%
\begin{mmdp}\label{dp:scp}{\bfseries Semantic compatibility principle}
\quad Semantic compatibility is a precursor to semantic interoperability. In order to consolidate semantic compatibility between collaborating software agents, both agents must confess to their ontological commitment underlying their domain models.\end{mmdp} 
&
\begin{description}[labelwidth=3.7cm,leftmargin=3.7cm+1ex,nosep,topsep=2ex,labelsep=1ex,font=\bfseries]
\item[Type of information:] Business, Data;
\item[Quality attributes:] Semantics, accuracy, interoperability, usability;
\end{description} \\
\begin{enumerate}[left=6pt, nosep]
  \item Following [@Grice:1991BT], it testifies to the manner of communication to avoid obscurity of expression and ambiguity;
  \item An ontological commitment represents a philosophical stance on fundamental categories and principles, which are foundational to semantics [@Brandt2021a, Sec.3.1];
  \item Each modelling language and, hence, every (semantic) model, carries an ontological commitment, at least implicitly;
  \item The semantic validity can be assessed only if the underlying ontological commitment can be referred to;
  \item Any assessment towards semantic interoperability of two semantic theories cannot be made without an assessment of the similarity between their underlying ontological commitments;
  \item The semantic meaning of data as specified by their SM are subordinate to the categorisation, scope and principles that the applied modelling language commits to; 
  \item Semantic compatibility is a necessary condition for the emergence of sIOP; 
  \item Without adherence to this principle, software agents will suffer from semantic incompatibility and therefore experience significant issues with their capability to establish sIOP in their communication with each other.
\end{enumerate}
&
\begin{enumerate}[left=10pt, nosep]
  \item Software agents that share the same ontological commitment will enjoy compatibility in the scope and nature of their semantics;
  \item Software agents that don’t share the same ontological commitment will know to take corrective actions to assure their semantic compatibility;
  \item By confessing to the ontological commitment underlying her domain model, a software agent realises the proper semantic environment to establish semantic interoperability with her.
\end{enumerate} \\
%
%
%
\begin{mmdp}\label{dp:alignment}{\bfseries Alignment principle:}\quad When a software agent engages in interoperation with (an)other software agent(s), establish for the exchanged data a maximal coherence between external semantic meaning and internal pragmatic meaning by formalising the alignment between the external and internal semantic meaning. \end{mmdp}
&
\begin{description}[labelwidth=3.7cm,leftmargin=3.7cm+1ex,nosep,topsep=2ex,labelsep=1ex,font=\bfseries]
  \item[Type of information:] Application, Data;
  \item[Quality attributes:] Semantics, semantic interoperability;
\end{description}\\
\begin{enumerate}[left=6pt, nosep]
  \item On processing external data, semantics manifest themselves as the reciprocity between data and processing code;
  \item Data are considered to carry the semantic meaning as specified by the Domain Model of the software agent;
  \item Formalising a correspondence relation between the semantic meanings of interoperating software agents effectively connects the external semantic meaning with the internal pragmatic meaning;
  \item By assuring that the internal semantic meaning encompasses the external semantic meaning, and by assuring that the semantic consequences of the latter extending the former are insignificant, collectively assures the semantic validity of the correspondence relation.
\end{enumerate}
&
\begin{enumerate}[left=10pt, nosep]
  \item The conversion from external to internal semantic meaning is specified by a correspondence;
  \item The collection of all correspondences specify the semantic alignment that holds between a pair of interoperating agents;
  \item Software agents that are unable to align their semantic meaning with the external semantic meaning cannot engage in sIOP without introducing phantom semantics, with unforeseen consequences in their data processing;
  \item Transcription of the semantic meaning from the DSP representation into the DSC representation is only dependent on the applied alignment language. A mediation component to perform such transcription, therefore, is in its operation agnostic to the representations of both the DSP and DSC. This turns the mediation component into a generic and potentially infrastructural one.
\end{enumerate} \\
%
%
%
\begin{mmdp}\label{dp:ssoc}{\bfseries Semantics/syntax separation principle:}\quad When a software agent engages in interoperation with (an)other software agent(s), resolve their semantic differences independently from the syntax of the exchanged data. \end{mmdp}
&
\begin{description}[labelwidth=3.7cm,leftmargin=3.7cm+1ex,nosep,topsep=2ex,labelsep=1ex,font=\bfseries]
\item[Type of information:] Data, Technology;
\item[Quality attributes:] semantic interoperability, portability, maintainability, efficiency, usability (reuse), reliability, functionality;
\end{description}
\\
\begin{enumerate}[left=6pt, nosep]
  \item Data schemata are defined to support the (de)serialisation processes that consolidate the data communications concern;
  \item Neglecting the principle of separation of concerns solidifies dependency between otherwise disjoint concerns, here the semantic level and the syntactic level of data communication;
  \item Access-and-play capabilities are supported by assuring minimal impact on software code when introducing semantic modifications;
  \item Minimising impact on software code that is concerned with data communication is realised by abstracting semantics away from the data schemata.
\end{enumerate}
&
\begin{enumerate}[left=10pt, nosep]
  \item Separation of concerns has a strong positive effect on software quality, including but not limited to sIOP;
  \item Removing any dependency between semantics and data syntax enables to support multiple communication paradigms without the need to modify the semantic representation;
  \item Similarly, modifications in the semantic representation, or supporting multiple semantic representations become possible without the need to modify the communication layer;
  \item Align semantics, not data schemata: Semantic reconciliation is applied at a higher conceptual level and abstracts away from data communication schemata;
  \item Heterogeneous semantics from multiple data sources are more easily supported;
  \item Semantic alignments imply the need for a mediation capability between the semantic representations of the communicating agents.
\end{enumerate} \\
%
%
%
\begin{mmdp}\label{dp:mediation}{\bfseries Semantic mediation principle:}\quad When software agents engage in interoperation, encapsulate the transcription between their semantic meaning.\end{mmdp}
&
\begin{description}[labelwidth=3.7cm,leftmargin=3.7cm+1ex,nosep,topsep=2ex,labelsep=1ex,font=\bfseries]
\item[Type of information:] Business, Data;
\item[Quality attributes:] Semantic interoperability, Consistency, reusability;
\end{description} \\
\begin{enumerate}[left=6pt, nosep]
  \item The semantic meaning is codified in (onto)logical representations;
  \item Keeping agents from referring to each others representation therefore requires transcription between representations;
  \item A solution where each agent needs to implement one transcription component between its own representation and each of its interoperating peer, increases complexity;
  \item Encapsulating the transcription into an alignment-based intermediary component results in less communication complexity and relieves the agents from development and maintenance of local wrappers;
\end{enumerate}
&
\begin{enumerate}[left=10pt, nosep]
  \item A mediator creates representational transparency between communicating agents, keeping agents from using each others representations;
  \item This enables independent development of the individual agent’s semantic meaning;
  \item The need to enforce a canonical semantic representation, viz. semantic homogeneity, expires, allowing semantic heterogeneity to become the norm;
  \item Each agent can reuse its semantic anchorage in any other interoperation;
  \item Data transcription logic can become a generic service provided by the communication infrastructure;
  \item Each agent can communicate with any other agent in its own native semantic representation.
\end{enumerate}\\
%
\bottomrule
\end{xltabular}
%%
%% END Design Principles Table
%%
%%%%%%%%%%%%%
