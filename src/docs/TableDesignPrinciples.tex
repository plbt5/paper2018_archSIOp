%%%%%%%%
%%
%% Tex code representing Design Principles Table
%%
\def\arraystretch{0} 
\begin{xltabular}[l]{\linewidth}{@{} >{\small\raggedright\arraybackslash}p{0.47\linewidth} >{\small\raggedright\arraybackslash}X @{}}

\caption{The sIOP design principles; structured according to \cite{Greefhorst2011} \label{tab:dps}} \\
\toprule
\multicolumn{1}{@{}l}{Design Principle} & \multicolumn{1}{l}{Dimensions \& Attributes} \\ 
\multicolumn{1}{l}{Rationale} & \multicolumn{1}{l}{\quad Implications} \\ \cmidrule(r){1-1} \cmidrule(l){2-2}
\endfirsthead
\multicolumn{2}{@{}l}{\ldots\ \tiny Continuation}\\ \midrule
\multicolumn{1}{@{}l}{Design Principle} & \multicolumn{1}{l}{Dimensions \& Attributes} \\ 
\multicolumn{1}{l}{Rationale} & \multicolumn{1}{l}{\quad Implications} \\ \cmidrule(r){1-1} \cmidrule(l){2-2}
\endhead
\midrule\multicolumn{2}{r@{}}{\tiny Continued \ldots}\\
\endfoot
\endlastfoot
%
%
%
%%%%%%%%%%%%%
%%
%% SEMANTIC HETEROGENEITY PRINCIPLE
%%
\begin{mmdp}\label{dp:shp}{\bfseries Semantic heterogeneity principle:}
\quad sIOP should strive to support multiple co-existing perspectives of use as opposed to enforcing one single perspective, and should be founded on the active reconciliation of semantic differences rather than on allowing one homogeneous semantic standard only.
\end{mmdp}
&
\begin{description}[labelwidth=3.7cm,leftmargin=3.7cm+1ex,nosep,topsep=2ex,labelsep=1ex,font=\bfseries]
  \item[Type of information:] business, data  
  \item[Quality attributes:] semantics, interoperability, reliability, reusability, efficiency
\end{description} \\
\begin{enumerate}[left=6pt, nosep]
  \item The user of the software agent has a natural, business-controlled perspective on the DoA;
  \item Data represent the state of affairs about that DoA, viewed from a particular perspective of use;
  \item Semantics of data, and with it the faithfulness with which the data refers to reality, is directly dependent on the perspective of use;
  \item Such view is just one particular perspective out of many equally legitimate ones;
  \item Equally legitimate perspectives on reality naturally create semantic heterogeneity;
  \item Accepting semantic heterogeneity implies the probable uniqueness of the agents view on reality;
  \item Without adherence to this principle, sIOP can be achieved only for dedicated DSC/DSP pairs, with collaborations that have been foreseen, for which the semantics are assumed stable over time, implemented with technology or platforms for which no or limited evolution is anticipated, and assuming that new collaborations won't emerge over time.
\end{enumerate}
&
\begin{enumerate}[left=10pt, nosep]
  \item Semantic heterogeneity allows software agents to maximise their semantic clarity, accuracy and, consequently, the faithfulness of its representation of the SoA in its DoA;
  \item Semantic heterogeneity allows software agents to be developed independently from each other, and particularly from potentially collaborative software agents;
  \item Semantic heterogeneity weakens the need to coordinate semantic specifications centrally; in stead, “anyone can say anything about any topic”, resulting in formalising the current state that semantic definitions are highly distributed;
  \item Maintenance and evolution of its semantics remain under the DSP’s own control throughout the whole lifespan of the agent, and only depends on its business interest for investment, no matter the growth and evolution of the domain;
  \item Semantic heterogeneity demands availability of semantic scalability solutions;
  \item Accepting semantic heterogeneity implies acceptance of semantic mediation and resulting semantic alignments in order to achieve sIOP.
\end{enumerate} \\
%
%
%
%%%%%%%%%%%%%
%%
%% SEMANTIC RESPONSIBILITY PRINCIPLE
%%
\begin{mmdp}\label{dp:srp}{\bfseries Semantic responsibility principle:}
\quad The software agent has the responsibility to explicate and disclose its OC, and, additionally, to disclose the semantic meaning of any exchanged data, to the extent necessary to allow comprehension of its data by the other involved agent(s). 
\end{mmdp}
&
\begin{description}[labelwidth=3.7cm,leftmargin=3.7cm+1ex,nosep,topsep=2ex,labelsep=1ex,font=\bfseries]
  \item[Type of information:] business, data;
  \item[Quality attributes:] semantics, interoperability, usability, efficiency;
\end{description} \\
\begin{enumerate}[left=6pt, nosep]
  \item Assume semantic heterogeneity to apply, allowing the software agents to maintain a particular perspective on the DoA;
  \item The DM mirrors the particular perspective taken, and applies a modelling language that expresses a particular OC;
  \item The extent to which collaborating agents' DMs are semantically compatible, is dependent on their applied OCs. 
  \item Data are formulated according to the DM's semantic meaning, and refer by means of it to the SoAs in the DoA;
  \item Grice’s maxims on communication, and particularly on serving the quantity and manner of communication, represent the natural constraints to respect. Consequently:
  \begin{enumerate}
    \item the responsibility to disclose the semantic meaning of the exchanged data can only fall with the DSP;
    \item both the DSP and DSC must apply and disclose their applied ontological commitment.
  \end{enumerate}
  \item Without adherence to this principle, the semantic meaning of the data expressed by the software agent can be considered flawed, inaccurate, incomplete, even unavailable or otherwise insufficient in its support for semantic interoperability.
\end{enumerate}
&
\begin{enumerate}[left=10pt, nosep]
  \item The domain model, which provides the semantic meaning to the data, is only dependent on the agent's own perspective on the application domain, and can therefore be designed independently from any other software agents;
  \item No matter the number of different communication peers, the software agent needs to specify the semantic meaning of its data elements only once in order to disclose it as often as necessary;
  \item By developing an explicit semantic specification of the data in advance, a software agent is prepared to engage in sIOP with other agents that can now connect to its semantics;
  \item By establishing collaboration with other DSPs or DSCs through its semantic meaning, the software agent ensures that the communication remains grounded in its own perspective.
\end{enumerate} \\
%
%
%
%%%%%%%%%%%%%%
%%
%% MINIMISE FAP PRINCIPLE
%%
\begin{mmdp}\label{dp:mfapp}{\bfseries Minimise FAP principle:}
\quad In order to minimise the False-Agreement Problem, the software agent's domain model must be constrained to minimize the difference between the semantic meaning that it allows and those that it intends.
\end{mmdp}
&
\begin{description}[labelwidth=3.7cm,leftmargin=3.7cm+1ex,nosep,topsep=2ex,labelsep=1ex,font=\bfseries]
  \item[Type of information:] business, data;
  \item[Quality attributes:] Semantics, interoperability, maturity;
\end{description} \\
\begin{enumerate}[left=6pt, nosep]
  \item The quintessential of exchanging data about the overlapping DoA is to indirectly exchange the SoA that the data refer to;
  \item Following \cite{Grice:1991BT}, it testifies to the quality of communication to take measures to prevent the exchange of SoAs that are invalid;
  \item The meaning that the agent's conceptualisation intends to convey is represented by a domain model. Such representation can only approximate the intended meaning. 
  \item Therefore, the SoAs that are allowed by the domain model do not necessarily represent the meaning that are intended by the conceptualisation. The difference between the allowed and intended meaning equals invalid meaning;
  \item Therefore, the agent should minimize the difference between the intended and the allowed meaning.
\end{enumerate}
&
\begin{enumerate}[left=10pt, nosep]
  \item When the allowed meaning of the domain model encompasses the intended meaning as close as possible:
  \begin{enumerate}
    \item the DSP allows minimal room to convey invalid semantic meaning;
    \item the DSC allows minimal room to comprehend the exchanged SoA invalidly;
  \end{enumerate}
  \item An overlap between the DSP's and DSC's domain models then also implies a maximal overlap in what is intended to convey with a minimal overlap of invalid meaning; 
  \item The false agreement problem is reduced to the minimal extent possible;
  \item Note that this applies for interoperability between agents as well as re-usability of the domain model by other agents;
\end{enumerate} \\
%
%
%
%%%%%%%%%%%%%%
%%
%% SEMANTIC PROTOCOL PRINCIPLE
%%
\begin{mmdp}\label{dp:spp}{\bfseries Semantic protocol principle:}
\quad The agent should provide the capability to respond to a semantic protocol that has the objective to prepare sIOP by conversing about semantics in order to consolidate the degree of semantic compatibility and the mutual comprehension of the exchanged data.
\end{mmdp}
&
\begin{description}[labelwidth=3.7cm,leftmargin=3.7cm+1ex,nosep,topsep=2ex,labelsep=1ex,font=\bfseries]
  \item[Type of information:] business;
  \item[Quality attributes:] semantics, interoperability, efficiency;
\end{description} \\
\begin{enumerate}[left=6pt, nosep]
  \item Following \cite{Grice:1991BT}, it testifies to the relation of communication to clarify the purpose of the communication;
  \item The essence of sIOP is to exchange SoAs about the DoA;
  \item Before SoAs can be exchanged, sIOP must be prepared by establishing agreement on the data's semantic meaning, demanding conversation about semantic in order to:
  \begin{enumerate}
    \item enable sIOP by establishing some level of semantic compatibility;
    \item determine the alignment(s) that is/are in demand;
    \item support the comprehension process.
  \end{enumerate}
  \item This implies to recognise the purpose and subject of communication, and hence, to maintain a shared vocabulary and communication process to denote them and switch between them, viz to rely on a standard semantic protocol.
\end{enumerate}
&
\begin{enumerate}[left=10pt, nosep]
  \item A semantic protocol can be considered a bootstrapping process, from establishing agreement on what protocol and version to use (if more than one is available), to being ready to engage in exchanging SoAs;
  \item A semantic protocol talks about semantic meaning and can therefore be considered to operate on a meta-semantic level, implying that it is independent from the particular semantics that apply to the exchanged SoA;
  \item This allows the semantic protocol to become an infrastructural service in support of sIOP.
\end{enumerate} \\
%
%
%
%%%%%%%%%%%%%%
%%
%% SEPARATE DATA PAYLOAD SYNTAX from SEMANTIC SYNTAX
%%
\begin{mmdp}\label{dp:sds-ss}{\bfseries Separate data syntax from semantics syntax principle:}
\quad Separate syntax representing the data payload in the agent interoperation from syntax representing the semantic payload of the data.
\end{mmdp}
&
\begin{description}[labelwidth=3.7cm,leftmargin=3.7cm+1ex,nosep,topsep=2ex,labelsep=1ex,font=\bfseries]
  \item[Type of information:] business, data;
  \item[Quality attributes:] semantics, interoperability, maintainability, efficiency;
\end{description} \\
\begin{enumerate}[left=6pt, nosep]
  \item Following \cite{Grice:1991BT}, it testifies to the relation of communication to clarify the purpose of the communication;
  \item The essence of sIOP is to exchange SoAs about the DoA;
  \item Before SoAs can be exchanged, sIOP must be prepared by establishing agreement on the data's semantic meaning, the domain models used, achieving semantic compatibility and the semantic topology that applies;
  \item This implies to recognise the purpose and subject of communication, and hence, to maintain a shared vocabulary and communication process to denote them and switch between them, viz to rely on a standard semantic protocol.
\end{enumerate}
&
\begin{enumerate}[left=10pt, nosep]
  \item A semantic protocol can be considered a bootstrapping process, from establishing agreement on what protocol and version to use (if more than one is available), to being ready to engage in exchanging SoAs;
  \item A semantic protocol talks about semantic meaning and can therefore be considered to operate on a meta-semantic level, implying that it is independent from the particular semantics that apply to the exchanged SoA;
  \item This allows the semantic protocol to become an infrastructural service in support of sIOP.
\end{enumerate} \\
%
%
%
%%%%%%%%%%%%%%
%%
%% SEMANTIC MODULARITY PRINCIPLE
%%
\begin{mmdp}\label{dp:smop}{\bfseries Semantic modularity principle:}
\quad When it is reasonable to expect that the domain model (DM) of the agent will grow throughout its lifetime, the DM should allow for modularisation methods that simplify and downsize the DM into manageable parts. 
\end{mmdp}
&
\begin{description}[labelwidth=3.7cm,leftmargin=3.7cm+1ex,nosep,topsep=2ex,labelsep=1ex,font=\bfseries]
  \item[Type of information:] business, data;
  \item[Quality attributes:] functionality (interoperability), efficiency (time behaviour, resource behaviour), maintainability (stability, changeability, manageability);
\end{description} \\
\begin{enumerate}[left=6pt, nosep]
  \item DMs have the tendency to grow. The larger the DM, the more complicated and the more difficult to keep its operational and maintenance traits under critical limits: the agent will be hampered by an increasingly unmanageable DM to the extent that the agent's operational deployment becomes impeded and/or its life time is significantly reduced;
  \item Modularising the DM into submodules allows for sizes that remain appropriate in terms of performance and maintenance;
  \item A modularisation method implements a mechanical approach towards either division into modules, or composition into the whole, according to the specific purpose as pursued by the particular modularisation.
\end{enumerate}
&
\begin{enumerate}[left=10pt, nosep]
  \item For each module, take measures to validate that it doesn't introduce (noticeable or relevant) differences in the outcome of the operation with respect to the whole
  \item Since a DM is self-contained and because a modularisation of a model doesn't require to adopt a physical divide, several modularisations can be maintained independently from each other, each designed for its own purpose, e.g., performance, scalability, governance;
  \item This allows for the capability to apply operations over generated modules, and to return their results back into the main repository.
\end{enumerate} \\
%
%
%
%%%%%%%%%%%%%
%%
%% SEMANTIC TRANSPARENCY PRINCIPLE
%%
\begin{mmdp}\label{dp:stp}{\bfseries Semantic transparency principle:}
\quad The software agent must remain agnostic to the local conceptualisation(s) and representation(s) of the agent(s) it collaborates with by implementing APIs that access semantic functionalities without committing to the particular domain semantics. Standardise communication services and their parameters on the semantic metalevel only, and only allow domain semantics as payload to the API.\end{mmdp}
&
\begin{description}[labelwidth=3.7cm,leftmargin=3.7cm+1ex,nosep,topsep=2ex,labelsep=1ex,font=\bfseries]
\item[Type of information:] Data, Technology;
\item[Quality attributes:] functionality (interoperability), portability, maintainability, efficiency, usability (reuse), reliability;
\end{description}
\\
\begin{enumerate}[left=6pt, nosep]
  \item Semantic heterogeneity is a matter of fact, and implies that agents apply different representations and conceptualisations on the DoA;
  \item These semantic differences must be reconciled without introducing interdependencies between the DM's of the DSP and DSC on syntax (semantic representation) and semantic principles (conceptualisation);
  \item Form and function of APIs remain an ICT responsibility, in support of the business collaboration, whereas form and function of the business collaboration remains a business responsibility;
  \item Changes to the ICT should be minimal when business collaboration and/or domain semantics are modified, and vice versa;
  \item Since the ICT discipline is not responsible for the domain semantics, implement communication services using meta-semantics that communicate \emph{about} semantics, encapsulating the exchange of domain semantics, only.
\end{enumerate}
&
\begin{enumerate}[left=10pt, nosep]
  \item Agents can communicate without syntactic or conceptual dependency, in their own local representations founded on their own local conceptualisation;
  \item Agents do not require re-implementation of their communication services when business or semantics change; vice-versa, technological changes or software modifications do not require change of the semantic and business levels;
  \item Governance and maintenance on semantics remain under one's own control, without demanding technological or software changes;
  \item sIOP is realised declaratively, improving consistency of semantics, and allowing infrastructural components for generic semantic services;
  \item The business/IT alignment improves through a disconnect between exchange of data and of semantics.
\end{enumerate} \\
%
%
%
%%%%%%%%%%%%%
%%
%% ALIGNMENT PRINCIPLE
%%
\begin{mmdp}\label{dp:ap}{\bfseries Alignment principle:}
\quad Align semantic meanings, not data schemata or data syntax.  \end{mmdp}
&
\begin{description}[labelwidth=3.7cm,leftmargin=3.7cm+1ex,nosep,topsep=2ex,labelsep=1ex,font=\bfseries]
  \item[Type of information:] Application, Data;
  \item[Quality attributes:] Semantics, semantic interoperability;
\end{description}\\
\begin{enumerate}[left=6pt, nosep]
  \item On processing (external) data, semantics manifest themselves as the reciprocity between semantic meaning (carried by data) and pragmatic meaning (carried by processing code);
  \item The purpose of semantic reconciliation is to establish semantic coherence between the DSP's semantic meaning and the DSC's pragmatic meaning, as specified by the DMs of the software agents;
  \item Formalising a correspondence relation between the semantic meanings of interoperating software agents effectively connects the external semantic meaning with the internal pragmatic meaning, providing semantic coherence to exist already by the DSC;
  \item Check the existence of one of the two sufficient conditions to consolidate the semantic validity of the correspondence relation: 
  \begin{enumerate}
    \item The internal (DSC's) semantic meaning encompasses the external (DSP's) semantic meaning, or
    \item The semantic consequences that result from the absence of the previous condition are (or can be made) insignificant for the DSC's existing semantic coherence.
  \end{enumerate}
\end{enumerate}
&
\begin{enumerate}[left=10pt, nosep]
  \item The conversion from external to internal semantic meaning is specified by a correspondence;
  \item The collection of all correspondences specify the semantic alignment that holds between a pair of interoperating agents;
  \item Software agents that are unable to align their semantic meaning with the external semantic meaning cannot engage in sIOP without introducing phantom semantics, with unforeseen consequences in their data processing;
  \item Despite the assumption that computers are token-based only without the capability to establish semantics, authoring the semantic alignment necessarily remains the \emph{only} human effort in establishing sIOP.
\end{enumerate} \\
%
%
%
%%%%%%%%%%%%%
%%
%% CORRESPONDENCE RELATION PRINCIPLE
%%
\begin{mmdp}\label{dp:crp}{\bfseries Correspondence relation principle:}
\quad A correspondence relation must preserve semantic meaning by addressing conceptual differences as well as convention-induced value-space differences. The alignment language must be expressive enough to represent those differences, the resolution of which is a necessary condition to achieve sIOP, providing faithful DMs from both collaborating agents. \end{mmdp}
&
\begin{description}[labelwidth=3.7cm,leftmargin=3.7cm+1ex,nosep,topsep=2ex,labelsep=1ex,font=\bfseries]
  \item[Type of information:] Data;
  \item[Quality attributes:] Semantics, semantic interoperability;
\end{description}\\
\begin{enumerate}[left=6pt, nosep]
  \item Assume the DSP and DSC each have their semantics faithfully specified by domain models;
  \item Assume that the DMs of the DSP and the DSC are compatible but express differences in their conceptualisation underlying their semantic meaning, resulting from alternate perspectives on the DoA;
  \item Then, assume we ignore:
  \begin{enumerate}
    \item the preservation of semantic meaning: then the DSC is allowed to ultimately refer to different individuals and, hence, a different SoA than specified by the DSP; 
    \item meta-properties of the relation between the concepts: then the DSC is allowed to infer SoAs that are not warranted by its premises;
    \item constraints to falsify possible interpretations: then the DSC allows SoA that are not intended by the DSP; 
    \item units of measurement: then the DSC allows the Mars Climate Orbiter to crash into the Martian atmosphere \cite{Leveson2004};
    \item lower value scale resolution: then the value masks distinctions that the DSC can comprehend as a different SoA than it will act upon;
    \item distinctions between ordinal, interval and ratio scales: then the DSC is allowed to make comparisons between incomparable values, resulting in invalid pragmatic meaning;
  \end{enumerate}
  \item Without addressing the conceptual and (convention-induced) value space differences, sIOP cannot emerge.
\end{enumerate}
&
\begin{enumerate}[left=10pt, nosep]
  \item With the exception of value space differences that originate from measurement system characteristics, convention-induced value space differences and conceptual differences can be specified such that preservation of semantic meaning can be guaranteed;
  \item Although correspondence relations are to be specified for each and every difference that applies in semantic meaning between the DSP and DSC, the modeling language to represent alignments remains agnostic to the specifics of the DMs involved (except for the representation of individuals in the DM).
\end{enumerate} \\
%
%
%
%%%%%%%%%%%%%
%%
%% HUMAN EFFICIENCY PRINCIPLE
%%
\begin{mmdp}\label{dp:hep}{\bfseries Human efficiency principle:}
\quad The human responsibility in semantic reconciliation should, to the extent possible, focus on auditing semantic alignments, and minimize their authoring. Maximise the use of reconciliation and auditing tools as well as the reuse of available alignment information.\end{mmdp}
&
\begin{description}[labelwidth=3.7cm,leftmargin=3.7cm+1ex,nosep,topsep=2ex,labelsep=1ex,font=\bfseries]
\item[Type of information:] Business, Data, Technology;
\item[Quality attributes:] semantic interoperability, maintainability, efficiency, usability (reuse);
\end{description}
\\
\begin{enumerate}[left=6pt, nosep]
  \item Computers cannot genuinely understand semantics, therefore, semantic reconciliation require human effort;
  \item Computer support can significantly improve human efficiency, hence, maximise the use of tools where possible, notably during reconciliation and authoring;
  \item Further efficiency can be gained by (re-)use of all available information, notably ontologies and their (example) data, already established alignments and shared vocabularies alike;
  \item Have the tools harness, to the maximal extent possible, the available information;
  \item Follow all previous DPs in order to maximally prepare the software agents towards sIOP.
\end{enumerate}
&
\begin{enumerate}[left=10pt, nosep]
  \item The higher the efficiency with which alignments between software agents' DMs can be achieved, the closer one gets to access-and-play sIOP;
  \item A reconciliation approach based on bilateral alignments only, produces the atomic element from which multi-lateral alignments can be constructed;
  \item A reconciliation approach based on bilateral alignments only, surfaces concerns on scalability and semantic governance for which solutions must be sought before sIOP can exist in multi-stakeholder collaborations;
  \item Semantic alignments direct the required syntactic transcriptions that apply at the communication layer, which can be resolved independently from the semantic meaning and the communication protocol that apply. This can, therefore, be pushed into the infrastructure.
\end{enumerate} \\
%
%
%
%%%%%%%%%%%%%%
%%
%% BUSINESS CONVERSATION PROTOCOL PRINCIPLE
%%
\begin{mmdp}\label{dp:bcpp}{\bfseries Business conversation protocol principle:}
\quad The agent should provide the capability to engage in a business conversation with participating agents, with the objective to jointly achieve a conversation state that is considered a(n intermediate,) necessary step towards the collaboration goal.
\end{mmdp}
&
\begin{description}[labelwidth=3.7cm,leftmargin=3.7cm+1ex,nosep,topsep=2ex,labelsep=1ex,font=\bfseries]
  \item[Type of information:] Business;
  \item[Quality attributes:] Semantics, interoperability;
\end{description} \\
\begin{enumerate}[left=6pt, nosep]
  \item Business collaboration is by definition a matter of establishing, in concert, a SoA that represents the collaboration goal of the stakeholders. Such collaboration goal is a causal and/or structural composition of separate contributions from individual stakeholders;
  \item Such contribution, expressed as the stakeholder's SoA, is produced by telic behaviour of a software agent on behalf of its stakeholder;
  \item In order to achieve a joint result between collaborating agents, coordination between their telic behaviours is required;
  \item Such coordination, either specified in advance or defined ad-hoc at run-time or both, requires a protocol to support collaborating agents to express demands about the dependency between SoAs that are produced by other agents in order to achieve a joint conversation state:
  \begin{enumerate}
    \item A stakeholder-internal SoA might be achievable only in dependency on one or more SoAs from one or more collaborating stakeholders, and/or
    \item A joint conversation state necessarily collects, causally and structurally, stakeholder-internal SoAs that are produced by distinct software agents.
  \end{enumerate}
\end{enumerate}
&
\begin{enumerate}[left=10pt, nosep]
  \item Providing the availability of a business conversation protocol, each individual agent is capable of initiating a business conversation that it deems necessary for achieving its own responsibility in the collaboration;
  \item Such business conversation protocol can rely on coordination and directions only, i.e., following a business choreography or business orchestration process, or a combination of both;
  \item This implies that business conversation protocol will converse \emph{about} the SoAs that are subject of such coordination, introducing the pragmatic (business) dimension as meta-level to the semantic dimension of the SoAs;
  \item This not only improves loose coupling between interoperating agents that act on behalf of stakeholders in their business collaboration, but consolidates a clear separation with the semantic meaning that is exchanged by the business conversation protocol.
\end{enumerate} \\
%
%
%
%%%%%%%%%%%%%
%%
%% SEMANTIC SEPARATION OF CONCERNS PRINCIPLE
%%
%\begin{mmdp}\label{dp:ssocp}{\bfseries Semantic separation of concerns principle:}
%\quad The software agent must provide support for loosely coupled semantics: data communication precedes data understanding which precedes their use. To this end, discern (i) syntax representing the data payload of the communication from syntax representing the semantic payload of the data; (ii) data comprehension behaviour; (iii) telic behaviour; (iv) business conversation services. A fifth concern separates the domain model in sub-domains that show high cohesion and low coupling. \end{mmdp}
%&
%\begin{description}[labelwidth=3.7cm,leftmargin=3.7cm+1ex,nosep,topsep=2ex,labelsep=1ex,font=\bfseries]
%\item[Type of information:] Data, Technology;
%\item[Quality attributes:] semantic interoperability, portability, maintainability, efficiency, usability (reuse), reliability, functionality;
%\end{description}
%\\
%Neglecting the principle of separation of concerns solidifies dependency between otherwise disjoint concerns, here the semantic level and the syntactic level of communication. The following complementary functionalities can be separated from each other:
%\begin{enumerate}[left=6pt, nosep]
%  \item Separate communication syntax from semantic representation:
%  \begin{enumerate}
%    \item The purpose of communication is to exchange data; the purpose of semantics is to refer to a SoA;
%    \item Both purposes are completely independent from each other, and both are under control of separate disciplines.
%  \end{enumerate}
%  \item Data comprehension reconstructs, for all semantic meaning that it receives, all semantic reciprocities that are involved by those ASMs that are addressed by the semantic meaning that the data represent.
%  \begin{enumerate}
%    \item This behaviour can only be implemented by the DSC, and must gets its foundation from \cref{dp:spp,dp:alignment,dp:alignment-language};
%    \item It will implement \cref{dp:spp} to request semantic meaning from other DSPs to fulfil its pragmatic behaviour required to achieve comprehension;
%    \item Comprehension behaviour is geared towards establishing pre-defined agent states, represented by conclusions drawn by the involved agent's ASMs, that are warranted by the observed SoAs.
%  \end{enumerate};
%  \item Telic behaviour produces a SoA in line with the agent's purpose;
%  \begin{enumerate}
%    \item Agent states that emerge from comprehension behaviour determine the actions that the agent will perform, its telic behaviour;
%    \item Agent telic behaviour is directed towards achieving its purpose, the result of which represents a particular SoA.
%  \end{enumerate}
%  \item Business conversation services produce the coordination between collaborating agents, necessary to produce a shared conversation state;
%  \begin{enumerate}
%    \item A business conversation can be expressed as a hierarchical cluster graph or lattice graph over the involved SoAs, specifying the causal and structural dependency between agent-local SoAs and their collectively composed conversation state;
%    \item Each cluster (node) in such specification identifies an agent's SoA, whereas each edge specifies its dependency on other agent's SoAs;
%    \item At any point in the conversation, the conversation protocol resembles the Happy Family card game \footnote{Happy Families (card game), https://en.wikipedia.org/wiki/Happy_Families, accessed Feb 2023}, showing how the protocol itself does not involve domain semantics other than exchanging requested SoA, i.e., semantic meaning.
%  \end{enumerate}
%  \item Separate the domain model in disjoint sub-domains
%  \begin{enumerate}
%    \item When the DM defines a high variety of subjects in the DoA without a clear separation between them, a small change in semantics will induce a proliferation of modifications in the DM;
%    \item Modularisation of the DM minimises the proliferation of a change in one module to other modules.
%  \end{enumerate}
%\end{enumerate}
%&
%\begin{enumerate}[left=10pt, nosep]
%  \item Separation of concerns has a strong positive effect on software quality, including but not limited to sIOP;
%  \item Removing any dependency between semantics and data syntax enables to support multiple communication paradigms without the need to modify the semantic representation;
%  \item Access-and-play capabilities are supported by assuring minimal impact on software code when introducing semantic modifications. Minimising impact on software code that is concerned with data communication is realised by abstracting semantics away from the data schemata;
%  \item Similarly, modification in the semantic representation, or supporting multiple semantic representations, becomes possible without the need to modify the communication layer;
%  \item Semantic separation of concerns allows to align semantics as opposed to data schemata resulting the semantic reconciliation to be applied at a higher conceptual level, abstracting away from data communication schemata;
%  \item Heterogeneous semantics from multiple data sources are more easily supported;
%  \item Abstracting semantic alignments and collaboration logic away from agents implies the need for a semantic mediation capability in between the collaborating agents.
%\end{enumerate} \\
%
%
%
%%%%%%%%%%%%%
%%
%% SEMANTIC PROFILES PRINCIPLE
%%
%\begin{mmdp}\label{dp:sprp}{\bfseries Semantic profiles principle:}
%\quad Apply different semantic profiles for different collaborations. \end{mmdp}
%&
%\begin{description}[labelwidth=3.7cm,leftmargin=3.7cm+1ex,nosep,topsep=2ex,labelsep=1ex,font=\bfseries]
%\item[Type of information:] Data, Technology;
%\item[Quality attributes:] semantic interoperability, portability, maintainability, efficiency, usability (reuse), reliability;
%\end{description}
%\\
%\begin{enumerate}[left=6pt, nosep]
%  \item ;
%  \item ;
%  \item .
%\end{enumerate}
%&
%\begin{enumerate}[left=10pt, nosep]
%  \item ;
%  \item ;
%  \item ;
%  \item .
%\end{enumerate} \\
%
%
%
%%%%%%%%%%%%%
%%
%% SEMANTIC MEDIATION PRINCIPLE
%%
\begin{mmdp}\label{dp:smep}{\bfseries Semantic mediation principle:}
\quad Software agents must encapsulate the transcription between their semantic meaning, as well as the coordination and control of their business collaboration.\end{mmdp}
&
\begin{description}[labelwidth=3.7cm,leftmargin=3.7cm+1ex,nosep,topsep=2ex,labelsep=1ex,font=\bfseries]
\item[Type of information:] Business, Data;
\item[Quality attributes:] Semantic interoperability, Consistency, reusability;
\end{description} \\
\begin{enumerate}[left=6pt, nosep]
  \item The semantic meaning is codified in ontological representations;
  \item Keeping agents from referring to each others representation therefore requires transcription between representations;
  \item A solution where each agent needs to implement one transcription component between its own representation and each of its interoperating peer, increases the complexity of collaboration and its management for each agent;
  \item Encapsulating the transcription into an alignment-based intermediary component results in less communication complexity and relieves the agents from development and maintenance of local wrappers;
  \item Similarly, because coordination and control of the business conversations
  \begin{enumerate}
    \item can be separated from semantic transcription, and
    \item will be different for each business collaboration, but can be specified upfront 
  \end{enumerate}
  \item their implementation can be encapsulated in a separate component.
\end{enumerate}
&
\begin{enumerate}[left=10pt, nosep]
  \item A mediator creates representational transparency between communicating agents, keeping agents from using each others representations. This enables independent development of the individual agent’s semantic meaning;
  \item The need to enforce a canonical semantic representation, viz. semantic homogeneity, expires, allowing semantic heterogeneity to become the norm. Consequently, each agent can reuse the semantic meaning and representation from its DM for every other interoperation;
  \item Transcription of the semantic meaning from the DSP representation and value space into the DSC representation and value space is only dependent on the applied alignment language and agnostic to any particular semantics assigned to it. A mediation component to perform such transcription, therefore, remains generically applicable for all domains. This provides the potential to turn the mediation component and the alignment language into infrastructural elements;
  \item Abstracting the coordination and control of the collaboration into a separate component greatly simplifies the individual agent's logic, since partaking in any collaboration only requires communication with the controlling component;
  \item Since business collaboration enacts upon exchanging semantic meaning between collaborating agents, combing the collaboration coordinating component and the semantic mediation brings advantages about semantic scalability as well as semantic modularisation of the agent's DM;
  \item Each agent can collaborate (with any other agent) by communicating with the mediating component only, and in its own native semantic representation;
  \item Any agent wishing to join a collaboration only requires a specification of its role in the collaborating process, its semantic alignments necessary to partake in the conversations, and adoption of the semantic protocol.
\end{enumerate}\\
%
\bottomrule
\end{xltabular}
%%
%% END Design Principles Table
%%
%%%%%%%%%%%%%


