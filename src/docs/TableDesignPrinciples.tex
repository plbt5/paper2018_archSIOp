%%%%%%%%
%%
%% Tex code representing Design Principles Table
%%
\def\arraystretch{0} 
\begin{xltabular}[l]{\linewidth}{@{} >{\small\raggedright\arraybackslash}p{0.47\linewidth} >{\small\raggedright\arraybackslash}X @{}}

\caption{The sIOP design principles; structured according to \cite{Greefhorst2011} \label{tab:dps}} \\
\toprule
\multicolumn{1}{@{}l}{Design Principle} & \multicolumn{1}{l}{Dimensions \& Attributes} \\ 
\multicolumn{1}{l}{Rationale} & \multicolumn{1}{l}{\quad Implications} \\ \cmidrule(r){1-1} \cmidrule(l){2-2}
\endfirsthead
\multicolumn{2}{@{}l}{\ldots\ \tiny Continuation}\\ \midrule
\multicolumn{1}{@{}l}{Design Principle} & \multicolumn{1}{l}{Dimensions \& Attributes} \\ 
\multicolumn{1}{l}{Rationale} & \multicolumn{1}{l}{\quad Implications} \\ \cmidrule(r){1-1} \cmidrule(l){2-2}
\endhead
\midrule\multicolumn{2}{r@{}}{\tiny Continued \ldots}\\
\endfoot
\endlastfoot
%
%
%
%%%%%%%%%%%%%
%%
%% SEMANTIC HETEROGENEITY PRINCIPLE
%%
\begin{mmdp}\label{dp:shp}{\bfseries Semantic heterogeneity principle:}
\quad sIOP should strive to support multiple co-existing perspectives of use as opposed to enforcing one single perspective, and should be founded on the active reconciliation of semantic differences rather than on allowing one homogeneous semantic standard only.
\end{mmdp}
&
\begin{description}[labelwidth=3.7cm,leftmargin=3.7cm+1ex,nosep,topsep=2ex,labelsep=1ex,font=\bfseries]
  \item[Type of information:] business, data  
  \item[Quality attributes:] semantics, interoperability, reliability, reusability, efficiency
\end{description} \\
\begin{enumerate}[left=6pt, nosep]
  \item The user of the software agent has a natural, business-controlled perspective on the DoA;
  \item Data represent the state of affairs about that DoA, viewed from a particular perspective of use;
  \item Semantics of data, and with it the faithfulness with which the data refers to reality, is directly dependent on the perspective of use;
  \item Such view is just one particular perspective out of many equally legitimate ones;
  \item Equally legitimate perspectives on reality naturally create semantic heterogeneity;
  \item Accepting semantic heterogeneity implies the probable uniqueness of the agents view on reality;
  \item Without adherence to this principle, sIOP can be achieved only for dedicated DSC/DSP pairs, with collaborations that have been foreseen, for which the semantics are assumed stable over time, implemented with technology or platforms for which no or limited evolution is anticipated, and assuming that new collaborations won't emerge over time.
\end{enumerate}
&
\begin{enumerate}[left=10pt, nosep]
  \item Semantic heterogeneity allows software agents to maximise their semantic clarity, accuracy and, consequently, the faithfulness of its representation of the SoA in its DoA;
  \item Semantic heterogeneity allows software agents to be developed independently from each other, and particularly from potentially collaborative software agents;
  \item Semantic heterogeneity weakens the need to coordinate semantic specifications centrally; in stead, “anyone can say anything about any topic”, resulting in formalising the current state that semantic definitions are highly distributed;
  \item Maintenance and evolution of its semantics remain under the DSP’s own control throughout the whole lifespan of the agent, and only depends on its business interest for investment, no matter the growth and evolution of the domain;
  \item Semantic heterogeneity demands availability of semantic scalability solutions;
  \item Accepting semantic heterogeneity implies acceptance of semantic mediation and resulting semantic alignments in order to achieve sIOP.
\end{enumerate} \\
%
%
%
%%%%%%%%%%%%%
%%
%% SEMANTIC RESPONSIBILITY PRINCIPLE
%%
\begin{mmdp}\label{dp:srp}{\bfseries Semantic responsibility principle:}
\quad When it is reasonable to expect that the software agent will be engaged in collaboration or otherwise will interoperate with (an)other software agent(s), it is the responsibility of the DSP to disclose the semantic model of its DM and the ontological commitment that it applies, to the extent necessary for comprehension by the DSC.
\end{mmdp}
&
\begin{description}[labelwidth=3.7cm,leftmargin=3.7cm+1ex,nosep,topsep=2ex,labelsep=1ex,font=\bfseries]
  \item[Type of information:] Business, Data;
  \item[Quality attributes:] Semantics, interoperability, usability, efficiency;
\end{description} \\
\begin{enumerate}[left=6pt, nosep]
  \item Assume semantic heterogeneity to apply, resulting for the software agent in a particular perspective on the DoA;
  \item The semantic model mirrors the particular perspective taken, and applies a modelling language that expresses a particular OC that fits the perspective best; 
  \item Data are formulated according to the particular semantic model, and refer by means of it, to the SoAs in the DoA;
  \item Grice’s maxims on communication, and particularly on serving the quantity and manner of communication, represent the natural constraints to respect. Consequently, the responsibility to disclose the semantic model about the exchanged data, including its applied ontological commitment, can only lay by the DSP;
  \item Without adherence to this principle, the meaning of the data expressed by the software agent can be considered flawed, inaccurate, incomplete, even unavailable or otherwise insufficient in its support for semantic interoperability.
\end{enumerate}
&
\begin{enumerate}[left=10pt, nosep]
  \item The semantic model, which provides the semantic meaning to the data, is only dependent on the agent's own perspective on the application domain, and can therefore be designed independently from any other software agents;
  \item No matter the number of different communication peers, the software agent needs to specify the semantics of its data elements only once;
  \item By developing an explicit semantic specification of the data in advance, a software agent is prepared to engage in sIOP with other agents that can now connect to its semantics;
  \item By establishing collaboration with other DSPs or DSCs through its semantic model, the software agent ensures that the communication remains grounded in its own perspective.
\end{enumerate} \\
%
%
%
%%%%%%%%%%%%%%
%%
%% MINIMISE FAP PRINCIPLE
%%
\begin{mmdp}\label{dp:mfapp}{\bfseries Minimise FAP principle:}
\quad When it is reasonable to expect that the software agent will be engaged in collaboration or otherwise will interoperate with (an)other software agent(s), in order to minimise the False-Agreement Problem, its semantic model must be constrained to minimize the difference between the semantic meaning that it allows and those that is intended.
\end{mmdp}
&
\begin{description}[labelwidth=3.7cm,leftmargin=3.7cm+1ex,nosep,topsep=2ex,labelsep=1ex,font=\bfseries]
  \item[Type of information:] Business;
  \item[Quality attributes:] Semantics, interoperability, maturity;
\end{description} \\
\begin{enumerate}[left=6pt, nosep]
  \item The quintessential of exchanging data about the overlapping DoA is to indirectly exchange the SoA that the data refer to;
  \item Following \cite{Grice:1991BT}, it testifies to the quality of communication to take measures to prevent the exchange of SoAs that are invalid;
  \item The meaning that the agent's conceptualisation intends to convey is represented by a semantic model. Such representation can only approximate the intended meaning. 
  \item Therefore, the SoAs that are allowed by the semantic model do not necessarily represent the meaning that are intended by the conceptualisation. The difference between the allowed and intended meaning equals invalid meaning;
  \item Therefore, the agent should minimize the difference between the intended and the allowed meaning.
\end{enumerate}
&
\begin{enumerate}[left=10pt, nosep]
  \item When the allowed meaning of the semantic model encompasses the intended meaning as close as possible:
  \begin{enumerate}
    \item the DSP allows minimal room to convey invalid semantic meaning;
    \item the DSC allows minimal room to interpret the exchanged SoA invalidly;
  \end{enumerate}
  \item An overlap between the DSP's and DSC's semantic models then also implies a maximal overlap in what is intended to convey with a minimal overlap of invalid meaning; 
  \item The false agreement problem is reduced to the minimal extent possible;
  \item Note that this applies for interoperability between agents as well as re-usability of the DM by other agents;
\end{enumerate} \\
%
%
%
%%%%%%%%%%%%%%
%%
%% SEMANTIC PROTOCOL PRINCIPLE
%%
\begin{mmdp}\label{dp:spp}{\bfseries Semantic protocol principle:}
\quad When it is reasonable to expect that the software agent will be engaged in collaboration or otherwise will interoperate with (an)other software agent(s), the agent should provide the capability to respond to a semantic protocol that has the objective to prepare sIOP by conversing about semantics in order to consolidate semantic compatibility and mutual comprehension of the exchanged data.
\end{mmdp}
&
\begin{description}[labelwidth=3.7cm,leftmargin=3.7cm+1ex,nosep,topsep=2ex,labelsep=1ex,font=\bfseries]
  \item[Type of information:] Business;
  \item[Quality attributes:] Semantics, interoperability, efficiency;
\end{description} \\
\begin{enumerate}[left=6pt, nosep]
  \item Following \cite{Grice:1991BT}, it testifies to the relation of communication to clarify the purpose of the communication;
  \item The essence of sIOP is to exchange SoAs about the DoA;
  \item Before SoAs can be exchanged, sIOP must be prepared by establishing agreement on the data's semantic meaning, the semantic models used, achieving semantic compatibility and the semantic topology that applies;
  \item This implies to recognise the purpose and subject of communication, and hence, to maintain a shared vocabulary and communication process to denote them and switch between them, viz to rely on a standard semantic protocol.
\end{enumerate}
&
\begin{enumerate}[left=10pt, nosep]
  \item A semantic protocol can be considered a bootstrapping process, from establishing agreement on what protocol and version to use (if more than one is available), to being ready to engage in exchanging SoAs;
  \item A semantic protocol talks about semantic meaning and can therefore be considered to operate on a meta-semantic level, implying that it is independent from the particular semantics that apply to the exchanged SoA;
  \item This allows the semantic protocol to become an infrastructural service in support of sIOP.
\end{enumerate} \\
%
%
%
%%%%%%%%%%%%%
%%
%% ALIGNMENT PRINCIPLE
%%
\begin{mmdp}\label{dp:alignment}{\bfseries Alignment principle:}
\quad Aligning semantic meanings suffices for re-establishing semantic coherence and, hence, enabling sIOP. \end{mmdp}
&
\begin{description}[labelwidth=3.7cm,leftmargin=3.7cm+1ex,nosep,topsep=2ex,labelsep=1ex,font=\bfseries]
  \item[Type of information:] Application, Data;
  \item[Quality attributes:] Semantics, semantic interoperability;
\end{description}\\
\begin{enumerate}[left=6pt, nosep]
  \item On processing (external) data, semantics manifest themselves as the reciprocity between semantic meaning (carried by data) and pragmatic meaning (carried by processing code);
  \item The purpose of an alignment is to re-establish semantic coherence between the DSP's semantic meaning and the DSC's pragmatic meaning, as specified by the DMs of the software agents;
  \item Formalising a correspondence relation between the semantic meanings of interoperating software agents effectively connects the external semantic meaning with the internal pragmatic meaning, providing the existence of semantic coherence by the DSC;
  \item By assuring that the internal semantic meaning encompasses the external semantic meaning, and by assuring that the semantic consequences of the latter extending the former are (or can be made) insignificant for the available semantic coherence, consolidates the semantic validity of the correspondence relation.
\end{enumerate}
&
\begin{enumerate}[left=10pt, nosep]
  \item The conversion from external to internal semantic meaning is specified by a correspondence;
  \item The collection of all correspondences specify the semantic alignment that holds between a pair of interoperating agents;
  \item Software agents that are unable to align their semantic meaning with the external semantic meaning cannot engage in sIOP without introducing phantom semantics, with unforeseen consequences in their data processing;
  \item Despite the assumption that computers are token-based only without the capability to establish semantics, authoring the semantic alignment necessarily remains the \emph{only} human effort in establishing sIOP.
\end{enumerate} \\
%
%
%
%%%%%%%%%%%%%
%%
%% CORRESPONDENCE RELATION PRINCIPLE
%%
\begin{mmdp}\label{dp:alignment-language}{\bfseries Correspondence relation principle:}
\quad A correspondence relation that preserves semantic meaning addresses conceptual differences as well as convention-induced value space differences, the resolution of which is a necessary condition to achieve sIOP (providing faithful DMs from both collaborating agents). \end{mmdp}
&
\begin{description}[labelwidth=3.7cm,leftmargin=3.7cm+1ex,nosep,topsep=2ex,labelsep=1ex,font=\bfseries]
  \item[Type of information:] Data;
  \item[Quality attributes:] Semantics, semantic interoperability;
\end{description}\\
\begin{enumerate}[left=6pt, nosep]
  \item Assume the DSP and DSC each have their semantics faithfully specified by domain models;
  \item Assume that the DMs of the DSP and the DSC are compatible but express differences in their conceptualisation underlying their semantic meaning, resulting from alternate perspectives on the DoA;
  \item Then, assume we ignore:
  \begin{enumerate}
    \item the preservation of semantic meaning: then the DSC is allowed to ultimately refer to different individuals and, hence, a different SoA than specified by the DSP; 
    \item meta-properties of the relation between the concepts: then the DSC is allowed to infer SoAs that are not warranted by its premises;
    \item constraints to falsify possible interpretations: then the DSC allows SoA that are not intended by the DSP; 
    \item units of measurement: then the DSC allows the Mars Climate Orbiter to crash into the Martian atmosphere \cite{Leveson2004};
    \item lower value scale resolution: then the value masks distinctions that the DSC can comprehend as a different SoA than it will act upon;
    \item distinctions between ordinal, interval and ratio scales: then the DSC is allowed to make comparisons between incomparable values, resulting in invalid pragmatic meaning;
  \end{enumerate}
  \item Without addressing the conceptual and (convention-induced) value space differences, sIOP cannot emerge.
\end{enumerate}
&
\begin{enumerate}[left=10pt, nosep]
  \item With the exception of value space differences that originate from measurement system characteristics, convention-induced value space differences and conceptual differences can be specified such that preservation of semantic meaning can be guaranteed;
  \item Although correspondence relations are to be specified for each and every difference that applies in semantic meaning between the DSP and DSC, the modeling language to represent alignments remains agnostic to the specifics of the DMs involved (except for the representation of individuals in the DM).
\end{enumerate} \\
%
%
%
%%%%%%%%%%%%%
%%
%% SEMANTIC SEPARATION OF CONCERNS PRINCIPLE
%%
\begin{mmdp}\label{dp:ssoc}{\bfseries Semantic separation of concerns principle:}
\quad When a software agent engages in interoperation with (an)other software agent(s), separate data communication services from data comprehension and business collaboration services, and, separate the domain model in areas that show high cohesion and low coupling. \end{mmdp}
&
\begin{description}[labelwidth=3.7cm,leftmargin=3.7cm+1ex,nosep,topsep=2ex,labelsep=1ex,font=\bfseries]
\item[Type of information:] Data, Technology;
\item[Quality attributes:] semantic interoperability, portability, maintainability, efficiency, usability (reuse), reliability, functionality;
\end{description}
\\
\begin{enumerate}[left=6pt, nosep]
  \item Data schemata are defined to support the (de)serialisation processes that consolidate the data communications concern;
  \item Neglecting the principle of separation of concerns solidifies dependency between otherwise disjoint concerns, here the semantic level and the syntactic level of data communication;
  \item Access-and-play capabilities are supported by assuring minimal impact on software code when introducing semantic modifications;
  \item Minimising impact on software code that is concerned with data communication is realised by abstracting semantics away from the data schemata;
  \item When the DM defines a high variety of subjects in the DoA without a clear separation between them, a small change in semantics will induce a proliferation of modifications in the DM;
  \item Modularisation of the DM minimises the proliferation of a change in one module to other modules.
\end{enumerate}
&
\begin{enumerate}[left=10pt, nosep]
  \item Separation of concerns has a strong positive effect on software quality, including but not limited to sIOP;
  \item Removing any dependency between semantics and data syntax enables to support multiple communication paradigms without the need to modify the semantic representation;
  \item Similarly, modifications in the semantic representation, or supporting multiple semantic representations become possible without the need to modify the communication layer;
  \item SoC allows to align semantics as opposed to data schemata resulting the semantic reconciliation to be applied at a higher conceptual level, abstracting away from data communication schemata;
  \item Heterogeneous semantics from multiple data sources are more easily supported;
  \item Semantic alignments imply the need for a mediation capability between the semantic representations of the communicating agents.
\end{enumerate} \\
%
%
%
%%%%%%%%%%%%%
%%
%% SEMANTIC TRANSPARENCY PRINCIPLE
%%
\begin{mmdp}\label{dp:st}{\bfseries Semantic transparency principle:}
\quad Remain agnostic to the local conceptualisation(s) and representation(s) of the agent(s) you collaborate with. \end{mmdp}
&
\begin{description}[labelwidth=3.7cm,leftmargin=3.7cm+1ex,nosep,topsep=2ex,labelsep=1ex,font=\bfseries]
\item[Type of information:] Data, Technology;
\item[Quality attributes:] semantic interoperability, portability, maintainability, efficiency, usability (reuse), reliability;
\end{description}
\\
\begin{enumerate}[left=6pt, nosep]
  \item Semantic heterogeneity is a matter of fact, and implies agents to apply different representations and conceptualisations on the DoA;
  \item For scalable sIOP to emerge these differences must be reconciled without introducing interdependencies between the DM's of the DSP and DSC;
  \item Remaining agnostic to local representations allows for sIOP without syntactic dependency;
  \item Remaining agnostic to local conceptualisations allows for sIOP without dependency on local semantic principles;
  \item Semantic differences, particularly about representation and conceptualisation, can then be resolved declaratively, defined only once for each DSP/DSC pair.
\end{enumerate}
&
\begin{enumerate}[left=10pt, nosep]
  \item Communicate without syntactic or conceptual dependency, in your own local representations founded on your own local conceptualisation;
  \item Governance and maintenance on semantics remain under one's own control;
  \item sIOP is realised declaratively, improving consistency of semantics, and allowing infrastructural components for generic semantic services;
  \item Semantic transparency enforces these semantic services, both infrastructural and agent-specific, to be \emph{about} semantics, demanding standardisation on the level of meta-semantics only.
\end{enumerate} \\
%
%
%
%%%%%%%%%%%%%
%%
%% SEMANTIC PROFILES PRINCIPLE
%%
\begin{mmdp}\label{dp:sprof}{\bfseries Semantic profiles principle:}
\quad Apply different semantic profiles for different collaborations. \end{mmdp}
&
\begin{description}[labelwidth=3.7cm,leftmargin=3.7cm+1ex,nosep,topsep=2ex,labelsep=1ex,font=\bfseries]
\item[Type of information:] Data, Technology;
\item[Quality attributes:] semantic interoperability, portability, maintainability, efficiency, usability (reuse), reliability;
\end{description}
\\
\begin{enumerate}[left=6pt, nosep]
  \item ;
  \item ;
  \item .
\end{enumerate}
&
\begin{enumerate}[left=10pt, nosep]
  \item ;
  \item ;
  \item ;
  \item .
\end{enumerate} \\
%
%
%
%%%%%%%%%%%%%
%%
%% SEMANTIC MEDIATION PRINCIPLE
%%
\begin{mmdp}\label{dp:mediation}{\bfseries Semantic mediation principle:}
\quad When software agents engage in interoperation, encapsulate the transcription between their semantic meaning.\end{mmdp}
&
\begin{description}[labelwidth=3.7cm,leftmargin=3.7cm+1ex,nosep,topsep=2ex,labelsep=1ex,font=\bfseries]
\item[Type of information:] Business, Data;
\item[Quality attributes:] Semantic interoperability, Consistency, reusability;
\end{description} \\
\begin{enumerate}[left=6pt, nosep]
  \item The semantic meaning is codified in ontological representations;
  \item Keeping agents from referring to each others representation therefore requires transcription between representations;
  \item A solution where each agent needs to implement one transcription component between its own representation and each of its interoperating peer, increases complexity;
  \item Encapsulating the transcription into an alignment-based intermediary component results in less communication complexity and relieves the agents from development and maintenance of local wrappers;
\end{enumerate}
&
\begin{enumerate}[left=10pt, nosep]
  \item A mediator creates representational transparency between communicating agents, keeping agents from using each others representations;
  \item This enables independent development of the individual agent’s semantic meaning;
  \item The need to enforce a canonical semantic representation, viz. semantic homogeneity, expires, allowing semantic heterogeneity to become the norm;
  \item Each agent can reuse its semantic anchorage in any other interoperation;;
  \item Transcription of the semantic meaning from the DSP representation and value space into the DSC representation and value space is only dependent on the applied alignment language and agnostic to any particular semantics assigned to it. A mediation component to perform such transcription, therefore, remains generically applicable for all domains. This provides the potential to turn the mediation component and the alignment language into infrastructural elements;
  \item Each agent can communicate with any other agent in its own native semantic representation.
\end{enumerate}\\
%
\bottomrule
\end{xltabular}
%%
%% END Design Principles Table
%%
%%%%%%%%%%%%%
