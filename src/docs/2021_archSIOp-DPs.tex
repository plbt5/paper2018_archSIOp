%% File: elsarticle.latex
%% Purpose: Support Elsevier's `elsarticle` package through pandocomatic.
%%          To that end turn the generic `elsarticle-template.tex` into
%%          a template that can be used in conjunction with
%%          pandocomatic and scrivomatic.
%% Authors: Ian Max Andolina, Paul Brandt
%% Date   : Nov. 11, 2021
%% 
%% This file is a manual integration of:
%%      *   elsarticle-template.tex (NOT the `elsarticle-template-*.tex` 
%%          examples that appear in the elsarticle template for these are  
%%          not comprehensive enough for a full article).
%%          Sourced from https://www.latextemplates.com/template/elsarticle-academic-journal
%%      * custom.latex, a minimal portion of it to make it compatible with 
%%          pandocomatic, and removing those parts that potentially 
%%          redefine `elsarticle` definitions.
%%          Sourced from https://github.com/iandol/dotpandoc/tree/master/templates
%%
%% Backwards compatibility with both files need to be managed manually.
%%  
%% Usage notes:
%%      1 - You'd want to apply this in combination with:
%%          * pandocomatic-elsarticle.yaml, or its contents integrated
%%              in any other pandocomatic.yaml file
%%          * (optional) addstyles.sty
%%      2 - Connect <your>.md with pandocomatic's `elsevier` template
%%          * just like any other regular pandocomatic template, hence
%%          * Insert in <your>.md's yaml block:
%%            pandocomatic_:
%%              use-template:  
%%                - elsevier  
%%      2 - Run pandocomatic with <your>.md text as follows:
%%          * pandocomatic -b -c ./pandocomatic-elsarticle.yaml ./<your>.md
%%            This will generate <your>.tex, hence continue with
%%          * latexmk -interaction=nonstopmode -f -pv -time -xelatex <your>.tex
%%            This will generate <your>.pdf
%%            
%%
%% Regarding elsarticle ::
%% ---------------------------------------------
%%
%% Copyright 2007-2020 Elsevier Ltd
%% 
%% It may be distributed under the conditions of the LaTeX Project Public
%% License, either version 1.2 of this license or (at your option) any
%% later version.  The latest version of this license is in
%%    http://www.latex-project.org/lppl.txt
%% and version 1.2 or later is part of all distributions of LaTeX
%% version 1999/12/01 or later.
%%  
%%
%% Regarding dotpandoc's custom.latex ::
%% ---------------------------------------------
%% The template custom.latex is a fairly complicated piece of work 
%% (thanks to Ian Max Andolina) to produce scientific articles in a simple
%% workflow with Scrivener, pandocomatic and scrivomatic. 
%% I've taken bits and pieces from it to make the elsarticle-template
%% produce a tex document by application of a pandocomatic.yaml configuration.
%% 
%% 
%%%%%%%%%%%%%

\documentclass[sort&compress,preprint,authoryear,3p,twocolumn]{elsarticle}
%%
%% !Todo!: Resolve the conflicting `&` in `sort&compress` when parameterized as
%%         `classoption` in pandocomatic-elsarticle.yaml configuration.
%%         Currently, this parameter is added here directly as opposed to
%%         being pandocomatically configured.
%% !Todo!: Check whether one- or two-column mode has been selected, and
%%         establish the need to redefine Figures/Tables as done below.

%%%%%%%%%%%%% REQUIRED PACKAGES
%% The following packages are required for this template to remain compatible 
%% (or at least effective) with the Elsevier template.

%% Supporting colored links in citations to the References section.
\usepackage{xcolor}

%% The amssymb package provides various useful mathematical symbols
%% whereas the amsthm package provides extended theorem environments
\usepackage{amsmath,amssymb}

%% The hyperref package is required since pandocmatic inserts \hypertarget
%% around section titles and other inter-article linking.
\usepackage[]{hyperref}
% Setup hyperref to color the links, and insert authors, title and keywords
% as meta-data to the pdf document properties.
\hypersetup{
  pdftitle={Consolidating Semantic Interoperability in Contemporary Software Architecture Paradigms},
  pdfauthor={Paul Brandt, Eric Grandry, Marten van Sinderen, Twan
Basten},
  pdflang={en-GB},
  pdfkeywords={semantic interoperability, software
architectures, semantics, interoperability, design principles},
  colorlinks=true,
  pdfcreator={Scrivomatic, Pandoc and LaTeX}
}
%%

%%
%% Elsevier bibliography styles
%% ----------------------------
%% 
\bibliographystyle{elsarticle-harv}
\setcitestyle{authoryear,open={(},close={)}} %Citation-related commands
%% Add extra options of natbib.sty, if any specified.

%%
%% !Todo!: Check the specifics of reference styles from elsarticle-template-*.tex
%%         as mentioned in https://support.stmdocs.in/wiki/index.php?title=Model-wise_bibliographic_style_files
%%         against the biboptions / bibliographicstyle settings from pandocomatic-elsarticle.yaml 
%% !ToDo!: Some of the bibliographicstyle settings require \usepackage{numcompress}. 
%%         Check whether this is included by this very template.
%%

%% 
%% Handle figures and tables in two-column mode
%% --------------------------------------------
%% In a two-column paper, figures and tables can become too small or overflow
%% the column width. In two-column mode we need to restore the original 
%% behaviour:
%% - For figures and tables, one needs to use the starred version * of these 
%%   environments. This floats the environment to the top/bottom of page over
%%   both columns.
%%   Source: https://tex.stackexchange.com/questions/89462/page-wide-table-in-two-column-mode
%%   Hence, redefine those environments:
\makeatletter
\renewenvironment{figure}{%
  \begin{figure*}
 }{% 
  \end{figure*}
 }
\makeatother
\makeatletter
\renewenvironment{table}{%
  \begin{table*}
 }{% 
  \end{table*}
 }
\makeatother
%%
%% - Longtable & xltabular do not work well with multicolumns. Therefore, 
%%   redefine these environments to enforce a single column mode before
%%   its definition, and restore the two-column mode afterwards:
\usepackage{stackengine}    % !!See Note (1) at the bottom!! 
\usepackage{xltabular}		% Include longtable with column specifier X as in tabularx
\usepackage{booktabs}		% To enhance the quality of tables in LaTeX. 
                            % Guidelines are given as to what constitutes a 
                            % good table in this context.
\usepackage{etoolbox}       % Used for the \BeforeBegin- & \AfterEndEnvironment
\BeforeBeginEnvironment{longtable}{%
    \onecolumn%
}
\AfterEndEnvironment{longtable}{%
    \twocolumn%
}
\AtBeginEnvironment{longtable}{%
    \scriptsize%
}
\BeforeBeginEnvironment{xltabular}{%
    \onecolumn%
}
\AfterEndEnvironment{xltabular}{%
    \twocolumn%
}
%%   This is not an optimal solution since it can result in pages around the
%%   longtables that are partly blank unintentionally. Other solutions to
%%   resolve this are welcome.
%%
%% !ToDo!: redefine simple tables, booktab and tabular.
%% !ToDo!: Make this part conditional on the documentclass-defined one- or
%%         two-column mode. That is, only include it when necessary.
%%

%%
%%%%%%%%%%%%% end REQUIRED PACKAGES

%%%%%%%%%%%%% OPTIONAL PACKAGES
%% The following packages are optional for this template, as
%% per the Elsevier template .

%% For including figures, graphicx.sty has been loaded in
%% elsarticle.cls. If you prefer to use the old commands
%% please give \usepackage{epsfig}

%% The lineno packages adds line numbers. Start line numbering with
%% \begin{linenumbers}, end it with \end{linenumbers}. Or switch it on
%% for the whole article with \linenumbers.
\usepackage{lineno}
\modulolinenumbers[5]

%% When you have an Orcid ID (refer to) you might want to include that
%% in your author-field as \orcidlink{<orcid>}
\usepackage{orcidlink}

\usepackage{blindtext}

%%
%% Some pandocomatic-specified options require additional packages
%%


%% Tightlist
\setlength{\emergencystretch}{3em} % prevent overfull lines
\providecommand{\tightlist}{%
  \setlength{\itemsep}{0pt}\setlength{\parskip}{0pt}}
%%

%%
%%%%%%%%%%%%% end OPTIONAL PACKAGES

%%%%%%%%%%%%% PROJECT SPECIFIC PACKAGES 
%%
%% These are managed through the `include-in-header:` parameter in the 
%% pandocomatic.yaml specification. Directly by listing the packages there, or 
%% indirectly by specifying one single file, e.g., `addstyles.sty`, that 
%% collects the required packages.
%%
%%%%%%%%%%%%% end PROJECT SPECIFIC PACKAGES

%%%%%%%%%%%%% DEPENDENCIES
%% When including packages by header-includes, 
%% dependencies might occur with them.
%% Resolve these dependencies below this line.
%%

%% Dependent on package{graphicx}
%%

%%
%%%%%%%%%%%%% end DEPENDENCIES

%%%%%%%%%%%%% ADD & PARAMETERIZE elsarticle template elements

\journal{Information and Software Technology}

\begin{document}

\begin{frontmatter}

%% TITLE elements
  \title{Consolidating Semantic Interoperability in Contemporary
Software Architecture Paradigms\tnoteref{id1,id2}}
  \tnotetext[id1]{version 2.0-0}\tnotetext[id2]{Tables \& figures with
2-columns incorrectly placed; will be corrected when submitting.}
%%

%% AUTHOR elements
\author[1]{Paul
Brandt\fnref{0000-0002-2353-967X}\corref{corrauth}}\fntext[0000-0002-2353-967X]{ORCID: \orcidlink{0000-0002-2353-967X}0000-0002-2353-967X}\ead{paul@brandt.name}\cortext[corrauth]{Corresponding author} \author[2]{Eric
Grandry}\ead{egrandry@gmail.com} \author[3]{Marten van
Sinderen}\ead{m.j.vansinderen@utwente.nl} \author[1]{Twan
Basten}\ead{a.a.basten@tue.nl}
%%

%% INSTITUTE elements
\affiliation[1]{organization={Eindhoven University of Technology (TU/e),
Eindhoven, The Netherlands}} \affiliation[2]{organization={Ministry of
Mobility and Public Works, Luxembourg,
Luxembourg}} \affiliation[3]{organization={University of Twente (UT),
Enschede, The Netherlands}} 
%%

%% ABSTRACT text

%%Graphical abstract
%\begin{graphicalabstract}
%\includegraphics{grabs}
%\end{graphicalabstract}

%%Research highlights
%\begin{highlights}
%\item Research highlight 1
%\item Research highlight 2
%\end{highlights}
%%

%% KEYWORDS elements
\begin{keyword}
semantic interoperability\sep software
architectures\sep semantics\sep interoperability\sep design principles
\end{keyword}
%%

\end{frontmatter}

%% Consider the use of line numbers
\linenumbers


%%%%%%%%%%%%% MAIN TEXT

\hypertarget{overview-of-design-principles}{%
\section{Overview of Design
Principles}\label{overview-of-design-principles}}

\hypertarget{dp-2.1}{%
\section{DP 2.1}\label{dp-2.1}}

\begin{mmdp}[The responsibility for the semantic meaning of data lays with the source]\label{dp:rfsm}

When it is reasonable to expect that the software agent will be engaged in collaboration or otherwise will interoperate with (an)other software agent(s), it is the responsibility of the software architect to serve the quantity and manner of the potential interoperability by specifying the semantics of the data in advance. 

\textbf{Type of information:} business, data  \\
\textbf{Quality attributes:} semantics, interoperability, usability, efficiency   \\
\textbf{Rationale:}
\begin{enumerate}
  \item Data represent the state of affairs of some part of the world, viewed from a particular perspective of use. Such view is just one particular perspective out of many equally legitimate ones;
  \item Semantic heterogeneity, a direct consequence of the equally legitimate perspectives on reality, should not be considered a bug to resolve, but a feature to preserve and nurture in order to maximise semantic clarity and accuracy;
  \item Accepting semantic heterogeneity implies the probable uniqueness of the agents view on reality;
  \item Computers are not capable of genuine understanding, hence cannot establish semantics from data and thus require the human-in-the-loop for that;
  \item The responsibility for formulating the semantics that are expressed by the data can only lay by the software architect that has taken the particular perspective on reality when carving out the entities of interest to the software application;
  \item On specifying semantics, Grice’s maxims on communication, and particularly on serving the quantity and manner of communication, represent the natural constraints to respect;
  \item Without adherence to this principle, the meaning of the data expressed by the software agent can be considered flawed, inaccurate, incomplete or otherwise insufficient in its support for semantic interoperability.
\end{enumerate}
\textbf{Implications:}
\begin{enumerate}
  \item The specification of the data semantics is only dependent on the agent’s own perspective on the application domain, and can therefore be fulfilled before any specific interoperability with communication peers;
  \item No matter the number of different communication peers, the software agent needs to have the semantics of its data elements specified only once;
  \item By providing an explicit semantic specification of the data, an agent facilitates other components and agents to connect to it and, consequently, grounds its semantic interoperability with them unequivocally.
\end{enumerate}  
\end{mmdp}

\hypertarget{dp-2.2}{%
\section{DP 2.2}\label{dp-2.2}}

\begin{mmdp}[Semantic compatibility is a precursor to semantic interoperability.]\label{dp:eoc}

In order to consolidate semantic compatibility between collaborating agents, both agents must confess to their ontological commitment underlying their domain models.

\textbf{Type of information:} business, data  \\
\textbf{Quality attributes:} semantics, interoperability, usability, functionality   \\
\textbf{Rationale:}
\begin{enumerate}
  \item Semantic compatibility is a necessary condition for the emergence of sIOP;
  \item An ontological commitment represent a philosophical stance on fundamental categories and principles, which are foundational to semantics [@Brandt2021a, Sec.3.1];
  \item Each modelling language and, hence, every (domain) model, carries an ontological commitment, at least implicitly;
  \item The semantics of data as is specified by their domain model is subordinate to the categorisation and scope that follow from the ontological commitment of the applied modelling language; 
  \item Following [@Grice:1991BT], it testifies to the manner of communication to avoid obscurity of expression and ambiguity; 
  \item Without adherence to this principle, agents will suffer from semantic incompatibility and therefore experience significant issues with their capability to establish sIOP in their communication with each other.
\end{enumerate}
\textbf{Implications:}
\begin{enumerate}
  \item Actors that share the same ontological commitment will enjoy compatibility in the scope and nature of their semantics;
  \item Actors that don’t share the same ontological commitment will know to take corrective actions to assure their semantic compatibility;
  \item By providing an explicit specification of the ontological commitment underlying her domain model, an agent realises the proper semantic environment to establish semantic interoperability with her.
\end{enumerate}  
\end{mmdp}

\hypertarget{dp-3.1}{%
\section{DP 3.1}\label{dp-3.1}}

\begin{mmdp}[Align the internal and external semantic meaning of the exchanged data]\label{dp:alignment}

When a software agent engages in interoperation with (an)other software agent(s), establish for the exchanged data a maximal coherence between external semantic meaning and internal pragmatic meaning by formalising the alignment between the external and internal semantic meaning.   

\textbf{Type of information:} application, data  \\
\textbf{Quality attributes:} semantics, semantic interoperability   \\
\textbf{Rationale:}
\begin{enumerate}
  \item On processing external data, semantics manifest themselves as the reciprocity between data and processing code;
  \item Data are considered to carry the semantic meaning as specified by the Domain Model of the software agent;
  \item Formalising a correspondence relation between the semantic meanings of interoperating software agents effectively connects the external semantic meaning with the internal pragmatic meaning;
  \item By assuring that the internal semantic meaning encompasses the external semantic meaning, and by assuring that the semantic consequences of the latter extending the former are insignificant, collectively assures the semantic validity of the correspondence relation.
\end{enumerate}
\textbf{Implications:}
\begin{enumerate}
  \item The conversion from external to internal semantic meaning is specified by a correspondence;
  \item The collection of all correspondences specify the semantic alignment that holds between a pair of interoperating agents;
  \item Software agents that are unable to align their semantic meaning with the external semantic meaning cannot engage in sIOP without introducing phantom semantics, with unforeseen consequences in their data processing.
\end{enumerate}  
\end{mmdp}

\hypertarget{dp-3.2}{%
\section{DP 3.2}\label{dp-3.2}}

\begin{mmdp}[Separate semantics from communication syntax]\label{dp:ssoc}

When a software agent engages in interoperation with (an)other software agent(s), resolve their semantic differences independently from the syntax of the exchanged data.   

\textbf{Type of information:} data, technology  \\
\textbf{Quality attributes:} semantic interoperability, portability, maintainability, efficiency, usability (reuse), reliability, functionality   \\
\textbf{Rationale:}
\begin{enumerate}
  \item Data schemata are defined to support the (de)serialisation processes that consolidate the data communications concern;
  \item Neglecting the principle of separation of concerns solidifies dependency between otherwise disjoint concerns, here the semantic level and the syntactic level of data communication;
  \item Access-and-play capabilities are supported by assuring minimal impact on software code when introducing semantic modifications;
  \item Minimising impact on software code that is concerned with data communication is realised by abstracting semantics away from the data schemata.
\end{enumerate}
\textbf{Implications:}
\begin{enumerate}
  \item Separation of concerns has a strong positive effect on software quality, including but not limited to sIOP;
  \item Removing any dependency between semantics and data syntax enables to support multiple communication paradigms without the need to modify the semantic abstraction;
  \item Similarly, modifications in the semantic representation, or supporting multiple semantic representations become possible without the need to modify the communication layer;
  \item Align semantics, not data schemata: Semantic reconciliation is applied at a higher conceptual level and abstracts away from data communication schemata;
  \item Heterogeneous semantics from multiple data sources are more easily supported;
  \item Semantic alignments imply the need for a mediation capability between the semantic representations of the communicating agents.
\end{enumerate}  
\end{mmdp}

\hypertarget{dp-4.1}{%
\section{DP 4.1}\label{dp-4.1}}

\begin{mmdp}[Encapsulate how agents exchange semantic meaning]\label{dp:mediation}

When software agents engage in interoperation, encapsulate how the representation of their semantic meaning should be transcribed without inducing phantom semantics.   

\textbf{Type of information:} business, data  \\
\textbf{Quality attributes:} semantic interoperability   \\
\textbf{Rationale:}
\begin{enumerate}
  \item The semantic meaning is codified in (onto)logical representations;
  \item Keeping agents from referring to each others representation therefore requires transcription between representations;
  \item A solution where each agent needs to implement one transcription component between its own representation and each of its interoperating peer, increases complexity;
  \item Encapsulating the transcription into an alignment-based intermediary component results in less communication complexity and relieves the agents from development and maintenance of local wrappers;
\end{enumerate}
\textbf{Implications:}
\begin{enumerate}
  \item A mediator creates representational transparency between communicating agents, keeping agents from using each others representations;
  \item This enables independent development of the individual agent’s semantic meaning;
  \item The need to enforce a canonical semantic representation, viz. semantic homogeneity, expires, allowing semantic heterogeneity to become the norm;
  \item Each agent can reuse its semantic anchorage in any other interoperation;
  \item Data transcription logic can become a generic service provided by the communication infrastructure;
  \item Each agent can communicate with any other agent in its own native semantic representation.
\end{enumerate}  
\end{mmdp}

%%%%%%%%%%%%% APPENDICES
%% !ToDo!: Implement appendices.
%%
%% The Appendices part is started with the command \appendix;
%% appendix sections are then done as normal sections, i.e.,
%% \appendix
%% \section{}
%% \label{}

%%%%%%%%%%%%% BIBLIOGRAPHY
%% If you have bibdatabase file and want bibtex to generate the
%% bibitems:
%% 1. Set the `bibliography` parameter in the pandocomatic configuration;
%% 2. Respect the related remarks from pandocomatic-elsarticle.template.
%%
\bibliography{src/bib/CitedByMe-2021-archSIOp.bib}
%%


\end{document}

%%
%% Note 1: For some reason that I cannot diagnose, the package `stackengine`
%%         is required with the `\longtable` command as it is being used in 
%%         elsarticle.latex; without its inclusion, the following error is
%%         thrown, referring to the first four lines after `\longtable`:
%% ! Undefined control sequence.
%% <argument> (\columnwidth - 6\tabcolsep ) * \real 
%%                                                  {0.2893}
%%         Anyone who can diagnose the source of this error, please inform me.
%%
%% End of file `elsarticle.latex'.
