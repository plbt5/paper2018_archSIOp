%% File: elsarticle.latex
%% Purpose: Support Elsevier's `elsarticle` package through pandocomatic.
%%          To that end turn the generic `elsarticle-template.tex` into
%%          a template that can be used in conjunction with
%%          pandocomatic and scrivomatic.
%% Authors: Ian Max Andolina, Paul Brandt
%% Date   : Nov. 11, 2021
%% 
%% This file is a manual integration of:
%%      *   elsarticle-template.tex (NOT the `elsarticle-template-*.tex` 
%%          examples that appear in the elsarticle template for these are  
%%          not comprehensive enough for a full article).
%%          Sourced from https://www.latextemplates.com/template/elsarticle-academic-journal
%%      * custom.latex, a minimal portion of it to make it compatible with 
%%          pandocomatic, and removing those parts that potentially 
%%          redefine `elsarticle` definitions.
%%          Sourced from https://github.com/iandol/dotpandoc/tree/master/templates
%%
%% Backwards compatibility with both files need to be managed manually.
%%  
%% Usage notes:
%%      1 - You'd want to apply this in combination with:
%%          * pandocomatic-elsarticle.yaml, or its contents integrated
%%              in any other pandocomatic.yaml file
%%          * (optional) addstyles.sty
%%      2 - Connect <your>.md with pandocomatic's `elsevier` template
%%          * just like any other regular pandocomatic template, hence
%%          * Insert in <your>.md's yaml block:
%%            pandocomatic_:
%%              use-template:  
%%                - elsevier  
%%      2 - Run pandocomatic with <your>.md text as follows:
%%          * pandocomatic -b -c ./pandocomatic-elsarticle.yaml ./<your>.md
%%            This will generate <your>.tex, hence continue with
%%          * latexmk -interaction=nonstopmode -f -pv -time -xelatex <your>.tex
%%            This will generate <your>.pdf
%%            
%%
%% Regarding elsarticle ::
%% ---------------------------------------------
%%
%% Copyright 2007-2020 Elsevier Ltd
%% 
%% It may be distributed under the conditions of the LaTeX Project Public
%% License, either version 1.2 of this license or (at your option) any
%% later version.  The latest version of this license is in
%%    http://www.latex-project.org/lppl.txt
%% and version 1.2 or later is part of all distributions of LaTeX
%% version 1999/12/01 or later.
%%  
%%
%% Regarding dotpandoc's custom.latex ::
%% ---------------------------------------------
%% The template custom.latex is a fairly complicated piece of work 
%% (thanks to Ian Max Andolina) to produce scientific articles in a simple
%% workflow with Scrivener, pandocomatic and scrivomatic. 
%% I've taken bits and pieces from it to make the elsarticle-template
%% produce a tex document by application of a pandocomatic.yaml configuration.
%% 
%% 
%%%%%%%%%%%%%

\documentclass[sort&compress,preprint,3p,authoryear,twocolumn]{elsarticle}
%%
%% !Todo!: Resolve the conflicting `&` in `sort&compress` when parameterized as
%%         `classoption` in pandocomatic-elsarticle.yaml configuration.
%%         Currently, this parameter is added here directly as opposed to
%%         being pandocomatically configured.
%% !Todo!: Check whether one- or two-column mode has been selected, and
%%         establish the need to redefine Figures/Tables as done below.

%%%%%%%%%%%%% REQUIRED PACKAGES
%% The following packages are required for this template to remain compatible 
%% (or at least effective) with the Elsevier template.

%% Supporting colored links in citations to the References section.
\usepackage{xcolor}

%% The amssymb package provides various useful mathematical symbols
%% whereas the amsthm package provides extended theorem environments
\usepackage{amsmath,amssymb}

%% The hyperref package is required since pandocmatic inserts \hypertarget
%% around section titles and other inter-article linking.
\usepackage[]{hyperref}
% Setup hyperref to color the links, and insert authors, title and keywords
% as meta-data to the pdf document properties.
\hypersetup{
  pdftitle={understanding precedes use: Consolidating semantic interoperability in software architectures},
  pdfauthor={Paul Brandt, Eric Grandry, Marten van Sinderen, Twan
Basten},
  pdflang={en-GB},
  pdfkeywords={semantic interoperability, software
architectures, semantics, interoperability, design principles},
  colorlinks=true,
  pdfcreator={Scrivomatic, Pandoc and LaTeX}
}
%%

%%
%% Elsevier bibliography styles
%% ----------------------------
%% Elsevier requires natbib. 
\usepackage[authoryear]{natbib}
\setcitestyle{authoryear,open={(},close={)}} %Citation-related commands; modify for your convenience.
\bibliographystyle{elsarticle-harv}
%% Add extra options of natbib.sty, if any specified.

%%
%% !Todo!: Check the specifics of reference styles from elsarticle-template-*.tex
%%         as mentioned in https://support.stmdocs.in/wiki/index.php?title=Model-wise_bibliographic_style_files
%%         against the biboptions / bibliographicstyle settings from pandocomatic-elsarticle.yaml 
%% !ToDo!: Some of the bibliographicstyle settings require \usepackage{numcompress}. 
%%         Check whether this is included by this very template.
%%

%% 
%% Handle figures and tables in two-column mode
%% --------------------------------------------
%% In a two-column paper, figures and tables can become too small or overflow
%% the column width. In two-column mode we need to restore the original 
%% behaviour:
%% - For figures and tables, one needs to use the starred version * of these 
%%   environments. This floats the environment to the top/bottom of page over
%%   both columns.
%%   Source: https://tex.stackexchange.com/questions/89462/page-wide-table-in-two-column-mode
%%   Hence, redefine those environments when 'twocolumn' has been specified
    
\makeatletter
\renewenvironment{figure}{%
  \begin{figure*}
 }{% 
  \end{figure*}
 }
\makeatother
\makeatletter
\renewenvironment{table}{%
  \begin{table*}
 }{% 
  \end{table*}
 }
\makeatother
%%
%% - Longtable & xltabular do not work well with multicolumns. Therefore, 
%%   redefine these environments to enforce a single column mode before
%%   its definition, and restore the two-column mode afterwards:
\usepackage{stackengine}    % !!See Note (1) at the bottom!! 
\usepackage{xltabular}		% Include longtable with column specifier X as in tabularx
\usepackage{booktabs}		% To enhance the quality of tables in LaTeX. 
                            % Guidelines are given as to what constitutes a 
                            % good table in this context.
\usepackage{etoolbox}       % Used for the \BeforeBegin- & \AfterEndEnvironment
\BeforeBeginEnvironment{longtable}{%
    \onecolumn%
}
\AfterEndEnvironment{longtable}{%
    \twocolumn%
}
\AtBeginEnvironment{longtable}{%
    \scriptsize%
}
\BeforeBeginEnvironment{xltabular}{%
    \onecolumn%
}
\AfterEndEnvironment{xltabular}{%
    \twocolumn%
}
%%   This is not an optimal solution since it can result in pages around the
%%   longtables that are partly blank unintentionally. Other solutions to
%%   resolve this are welcome.
%%
%% !ToDo!: redefine simple tables, booktab and tabular.
%% !ToDo!: Make this part conditional on the documentclass-defined one- or
%%         two-column mode. That is, only include it when necessary.
%%


%%
%%%%%%%%%%%%% end REQUIRED PACKAGES

%%%%%%%%%%%%% OPTIONAL PACKAGES
%% The following packages are optional for this template, as
%% per the Elsevier template .

%% For including figures, graphicx.sty has been loaded in
%% elsarticle.cls. If you prefer to use the old commands
%% please give \usepackage{epsfig}


%% When you have an Orcid ID (refer to) you might want to include that
%% in your author-field as \orcidlink{<orcid>}
\usepackage{orcidlink}

\usepackage{blindtext}

%%
%% Some pandocomatic-specified options require additional packages
%%


%% Tightlist
\setlength{\emergencystretch}{3em} % prevent overfull lines
\providecommand{\tightlist}{%
  \setlength{\itemsep}{0pt}\setlength{\parskip}{0pt}}
%%

%%
%%%%%%%%%%%%% end OPTIONAL PACKAGES

%%%%%%%%%%%%% PROJECT SPECIFIC PACKAGES 
%%
%% These are managed through the `include-in-header:` parameter in the 
%% pandocomatic.yaml specification. Directly by listing the packages there, or 
%% indirectly by specifying one single file, e.g., `addstyles.sty`, that 
%% collects the required packages.
%% File   : addstyles.sty
%% Purpose: To include project specific packages without the need to insert
%%          multiplpe `header-includes:` statements in the 
%%          pandocomatic.yaml configuration file. 
%%          Although there is no functional difference between both approaches,
%%          for me this file seems to provide me with a more simple overview
%%          about which packages and their whereabouts. 
%%          Note that for this approach to work, you need to insert the name
%%          of this file in the `header-includes:` statement.     
%% Author : Paul Brandt, date Nov. 11, 2021
%%%%%%%%%%%
%
% Add your .sty files below
%
%   One can make a differentiation in where to store specific addons to your
%   LaTeX environment:
%   - Generics, i.e., re-usable ones, are maintained in folders below your TeX  
%     home directory. Use `kpsewhich -var-value=TEXMFHOME` at the command prompt 
%     to find your TeX Home. But respect the TeX Directory Structure (TDS) from
%     TeX Home downwards.
%   - Project specific stuff can be maintained in the same folder as your 
%     document for this project. 
%   No particular assumptions are made by the elsarticle templates with regard to 
%   locations of files.
%
%%%%%%%%%%%

\usepackage{mySemantics}
\usepackage{myTables}
\usepackage{myFigures}
%\usepackage{myGenerics}
\usepackage{inlineStackedConditional}

\usepackage[textsize=tiny,colorinlistoftodos]{todonotes}	
							% Creating proper todo bubbles adjacent to text. Refer to http://ctan.triasinformatica.nl/macros/latex/contrib/todonotes/todonotes.pdf

\usepackage{pdflscape}      % Extends package <lscape> to add PDF support to the environment `landscape`.  
                            % Applied for table on design Principles

\usepackage{afterpage}      % Causes the commands specified in its argument to be expanded after the current 
                            % page is output. The current page will be filled up with text after \afterpage{...}.
                            % Applied for table on design Principles

\usepackage{enumitem}		% List manipulation. Customize the three basic list environments 
                            % (enumerate, itemize and description) and design your own lists, 
                            % with a〈key〉=〈value〉syntax.
                            % Applied for table on design Principles

\usepackage{gensymb}        % The gensymb package provides a number of ‘generic’ macros, 
                            % which produce the same output in text and math mode. 
                            % Used for \celsius
                            
\usepackage{cleveref}		% sourced from http://tug.ctan.org/macros/latex/contrib/cleveref/cleveref.pdf 
                            % Makes varioref clever, and reuses its work to improve its own workings. 
							% The cleveref package must be loaded after all other packages that 
                            % don’t specifically support it. Therefore, to be safe, we declare it as last 
                            % in the document's preamble.
							% Also note that all \newtheorem definitions must be placed after the 
                            % cleveref package is loaded.
\usepackage{myTheorems} 
                            
\usepackage{MnSymbol}       % To create the \righthalfcap as Not-symbol

%%%%%%%%
%% Project specific stuff
%%%%%%%%

%% Make counter/subcounter pair for the Concern Theorem, such that we can number a Concern S.1
\newcounter{mccntr}			% alphabetical counter for main concern (S, R, M and C)
\renewcommand{\themccntr}{\Alph{mccntr}}
\newcounter{snc}[mccntr]		% arabic subcounter for the actual (sub)concerns
\renewcommand{\thesnc}{\themccntr\arabic{snc}}
\newcounter{ssnc}[snc]		% arabic subsubcounter where necessary
\renewcommand{\thessnc}{\themccntr\arabic{snc}.\arabic{ssnc}}


\theoremsymbol{}				% A Concern has no symbol, and
\theoremheaderfont{\bfseries}% has bold header, and
\theoremstyle{break}			% is based on break style, and
\renewtheorem{mmconcern}[snc]{Concern}	% counts with the above counter


%%%%%%%%%%%
%% End of File: addstyles.sty
%%%%%%%%%%%

%%
%%%%%%%%%%%%% end PROJECT SPECIFIC PACKAGES

%%%%%%%%%%%%% DEPENDENCIES
%% When including packages by header-includes, 
%% dependencies might occur with them.
%% Resolve these dependencies below this line.
%%

%% Dependent on package{graphicx}
%%

%%
%%%%%%%%%%%%% end DEPENDENCIES

%%%%%%%%%%%%% ADD & PARAMETERIZE elsarticle template elements

\journal{Information and Software Technology}

\begin{document}

\begin{frontmatter}

%% TITLE elements
  \title{understanding precedes use: Consolidating semantic
interoperability in software architectures\tnoteref{version}}
  \tnotetext[version]{version v1.1-45, 02/11/2023}
%%

%% AUTHOR elements
  \author[1]{Paul
Brandt\fnref{0000-0002-2353-967X}\corref{corrauth}}\fntext[0000-0002-2353-967X]{ORCID: \orcidlink{0000-0002-2353-967X}0000-0002-2353-967X}\ead{p.brandt@tue.nl}\cortext[corrauth]{Corresponding author}   \author[2]{Eric
Grandry\fnref{0000-0003-3553-8460}}\fntext[0000-0003-3553-8460]{ORCID: \orcidlink{0000-0003-3553-8460}0000-0003-3553-8460}\ead{egrandry@gmail.com}   \author[4]{Marten
van
Sinderen\fnref{0000-0001-7118-1353}}\fntext[0000-0001-7118-1353]{ORCID: \orcidlink{0000-0001-7118-1353}0000-0001-7118-1353}\ead{m.j.vansinderen@utwente.nl}   \author[1,3]{Twan
Basten\fnref{0000-0002-2274-7274}}\fntext[0000-0002-2274-7274]{ORCID: \orcidlink{0000-0002-2274-7274}0000-0002-2274-7274}\ead{a.a.basten@tue.nl}%%

%% INSTITUTE elements
\affiliation[1]{organization={Eindhoven University of Technology (TU/e),
Eindhoven, The Netherlands}} \affiliation[2]{organization={Ministry of
Mobility and Public Works, Luxembourg,
Luxembourg}} \affiliation[4]{organization={University of Twente (UT),
Enschede, The Netherlands}} \affiliation[3]{organization={ESI (TNO),
Eindhoven, The Netherlands}} 
%%

%% ABSTRACT text

%%Graphical abstract
%\begin{graphicalabstract}
%\includegraphics{grabs}
%\end{graphicalabstract}

%%Research highlights
%\begin{highlights}
%\item Research highlight 1
%\item Research highlight 2
%\end{highlights}
%%

%% KEYWORDS elements
\begin{keyword}
semantic interoperability\sep software
architectures\sep semantics\sep interoperability\sep design principles
\end{keyword}
%%

\end{frontmatter}

%% Consider the use of line numbers


%%%%%%%%%%%%% MAIN TEXT

\section{sIOP concerns}\label{siop-concerns}

\begin{verbatim}
Created:    6 April 2023 at 17:02
Modified:   7 June 2023 at 10:08
Status: Done
Voortgang:  Revised draft
\end{verbatim}

Present the problem of sIOP, show that it is an interesting/relevant
problem, describe our contributions to solving the problem, and present
the applied method / structure. of paper. As follows:

\emph{Purpose of sIOP}

When agents collaborate, data are being exchanged. Data carry semantic
meaning, implicitly. Semantic meaning is to be understood as referring
to a particular SoA in the DoA. Semantic interoperability refers to the
capability of collaborating software agents that allows the DSC to
comprehend what SoA the DSP is referring to with the exchanged data.
This is necessary for the DSC in order to use that data faithfully:
understanding precedes use.

\emph{Problem of sIOP}

When collaborating agents share the same semantic representation,
denoted as semantic homogeneity, they share equivalent semantic meaning,
allowing for an immediate data understanding that are ready to use. The
more successful this semantic standard, the more it will be used, the
larger the community of use, and the more it evolves to support
alternative or extended collaborative use cases. This requires stricter
and more extensive governance of the standard that soon impedes the
demands of the business to allow for dynamic collaborations with quick
response to business opportunities.

This decline in the purposeful application of semantic standards is a
general characteristic, which can be observed independently from the
domain or advances in support to governance of semantic standards.

\emph{Purpose of paper}

We take an alternative approach, away from semantic homogeneity and
investigate sIOP from the perspective of semantic heterogeneity. With
the latter we refer to a situation where no prior agreements are made by
the community towards the representation of domain semantics. This is an
interesting and relevant approach since a successful outcome will evade
the aforementioned problems and provide support to dynamic business
collaboration. The purpose of this paper is to identify the
architectural concerns about sIOP that are caused by semantic
heterogeneity, find design principles to their resolution, and
consolidate these in the view-based architecture paradigm.

\emph{Overview of paper}

!{[}Story board in terms of Concerns and Design
Principles.{]}{[}def:sIOPcore{]}

We identify four sets of concerns in relation to semantic
interoperability, three of which are of functional nature and one of
extra-functional nature. The first three main sections address the
functional concerns (semantic concerns, reconciliation concerns, and
mediation concerns); the fourth main section addresses extra-functional
concerns about semantic scalability. For each of the identified concern,
one or more design principles are defined. Finally, we present how these
DPs can be operationalised by their inclusion in a method in support of
view-based architectures.

\section{Semantic concerns}\label{semantic-concerns}

\begin{verbatim}
Created:    6 April 2023 at 17:07
Modified:   1 June 2023 at 17:39
Status: Done
Voortgang:  Revised draft
\end{verbatim}

We set the scene for semantic interoperability, and identify three
pivotal aspects to consider.

First, we use the term \emph{semantics} to refer to real-world semantics
as used and interpreted by humans. Since software is incapable of
genuine understanding, RWS cannot exist in software. Nevertheless,
software agents exchange data that carry semantic meaning and with that
refer to states of affairs in the domain. When processing such data,
they must be processed isomorphic to reality in order to remain faithful
to reality. This has been described in \citep{Brandt2021a}, and we
summarise our position towards semantics in software.

Second, we investigate the dynamic business collaboration concern, and
argue that its \emph{dynamic} characteristic will be difficult to
achieve on the basis of semantic homogeneity, which is the current
approach to achieve sIOP through (syntactic) standards that consolidate
semantics. Therefore, we consider semantic heterogeneity a necessary
condition in order to achieve sIOP in dynamic environments.

Third, a conversation between two human agents that share a mutual
purpose brings about a natural responsibility for both of them in order
to achieve that purpose. We cannot find an argument that suggests that
these responsibilities do not apply for collaborating software agents as
well. Hence, it is necessary to surface semantic concerns that follow
from the responsibilities that apply between communicating agents.

These three aspects set the environment in which sIOP for dynamic
business collaboration must thrive. We address the latter two concerns
only, since we consider the former concern to represent the software
engineering reality that cannot be changed.

\subsection{Semantics in software}\label{semantics-in-software}

\begin{verbatim}
Created:    6 April 2023 at 17:07
Modified:   25 May 2023 at 15:52
Status: Done
Voortgang:  Revised draft
\end{verbatim}

We summarise our position towards semantics in software from
\citep{Brandt2021a}. Main points that serve as foundation for this paper
are:

\begin{itemize}
\tightlist
\item
  Data carry semantic meaning, i.e., the notion of something by
  referring to that something as it appears in reality of the DoA.
\item
  Data processing carries pragmatic meaning, i.e., data produced through
  a mathematical process, warranted by its premises (the semantic
  meaning). Pragmatic meaning is contextual, purposeful, and refers to
  SoA in the DoA.
\item
  The Domain Model (DM) is allocated to address the semantic concerns.
  To that end, it contains semantic and pragmatic meaning, and by doing
  so represents the agent's (subjective) reflection of the SoAs that are
  relevant to it.
\item
  The System Model (SM) is allocated to address the concerns about
  fulfilling he agents purpose. It implements specific agent behaviour
  and contains agent internal system states. The SM has the potential to
  change SoAs.
\item
  The DM and the SM are disjoint, although the SM is semantically
  grounded in the DM.
\item
  Separate comprehensive behaviour from telic behaviour: comprehension
  leads to pragmatic meaning, telic behaviour leads to predefined,
  designed effects.
\end{itemize}

\subsection{Dynamic business concern}\label{dynamic-business-concern}

\begin{verbatim}
Created:    6 April 2023 at 17:07
Modified:   1 June 2023 at 18:30
Status: Incomplete
Voortgang:  Revised draft
\end{verbatim}

Business collaboration is driven by a mutual interest of different
stakeholders, where software agents act on behalf of the stakeholder by
exchanging semantic meaning about the DoA. Business collaboration
requires agreements to be made between stakeholders about the semantic
meaning that is carried by the data that the software agents exchange.
\emph{Dynamic} business collaboration is not different from that, except
for the fact that variations occur over time about the mutual interests,
or new or evolving business demands, involved stakeholders, and more.
The nature of the dynamics in business collaboration can be strategical,
tactical or ad-hoc, but almost always imply changes in the current
semantic agreements that affect few or many involved stakeholders. A
sudden increase in semantic heterogeneity will occur, and with it sIOP
will be impeded on precisely the collaboration of interest. The IT must
be aligned with the business again, and moreover, turn into effect
almost immediately. Equally important, standing business collaboration
must continue.

\begin{mmconcern}[Dynamic collaboration concern]\label{cncrn:dynamic-collaboration}
This concern reflects the tension that exists in dynamic business collaborations between the business desire to immediately engage in business collaboration, particularly on new business opportunities, and the community-wide distributed IT efforts required to achieve sIOP. The latter involves reconciling semantic differences, implementing the newly established semantics, and coordinating its deployment to all involved software agents. The concern deepens due to the distribution of responsibilities involved: the business/domain is responsible for the semantics, whereas the local IT is responsible for consolidating the alignment of the exchange of data and their processing by their software agents with the specified semantics.
\end{mmconcern}

Applied DPs: (INCOMPLETE with respect to concern)

\begin{itemize}
\tightlist
\item
  DP1: Semantic Heterogeneity Principle
\item
  DP5: Separate data syntax from semantics syntax principle
\end{itemize}

\subsection{Cooperative concern}\label{cooperative-concern}

\begin{verbatim}
Created:    6 April 2023 at 17:07
Modified:   25 May 2023 at 17:20
Status: In Review
Voortgang:  Revised draft
\end{verbatim}

\subsubsection{Semantic sufficiency}\label{semantic-sufficiency}

\begin{verbatim}
Created:    6 April 2023 at 17:07
Modified:   15 June 2023 at 09:41
Status: In Review
Voortgang:  1st draft
\end{verbatim}

What precise information is sufficient to provide to whom for the latter
to proceed?

The ability to establish what can be considered \emph{sufficient}
information depends on:

\begin{itemize}
\tightlist
\item
  Purpose of conversation, deferred to \cref{conversation-purpose}
\item
  Knowledge about what information is already available, what has been
  provided already, inferring what is missing
\item
  Individual agents in the conversation and their capabilities
\item
  Minimal
\end{itemize}

Maxim 1, the quantity maxim (``Make your contributions as informative as
is required (for the current purpose of the exchange), and not more than
is required''), refers to the purpose of the conversation, which we
investigate in \cref{conversation-purpose}. It also refers to the
semantic meaning that is exchanged with the data. Assume the DSP agent
to exchange that ``Mrs.~McNabb has got two sons''. The DSC can now
comprehend from this that Mrs.~McNabb has got exactly two sons. Without
application of this maxim, an ambiguity emerges because the DSC can then
also comprehend that Mrs.~McNabb has got at least two sons. Another
reason for the application of this maxim is about enabling the DSP to
know what the DSC agent knows about the SoAs. Based on this knowledge,
the scope of the conversation, viz what semantic meaning is sufficient
for the DSP to exchange, can be minimised to what it considers unknown
to the DSC. An efficient conversation \citep{Diggelen:2007vd} with the
potential for agents to digress into subdialogues about the SoAs
\citep{Engelmann2023} will emerge from that.

\begin{mmconcern}[Semantic sufficiency concern]\label{cncrn:qtymc}
An efficient and unambiguous agent discourse requires agents to be aware of each other and each other’s purposes and knowledge. Similarly, collaborative, task-related discourses require composition of knowledge about each other’s capabilities and individual knowledge about the SoAs that apply in the shared domain. This is a distributed capability that requires clear scoping and timing to prevent an overload of reasoning and irrelevant data exchange.
\end{mmconcern}

Applied DP:

\begin{itemize}
\tightlist
\item
  DPy: Support multi-agent belief model
\end{itemize}

\subsubsection{Semantic faithfulness}\label{semantic-faithfulness}

\begin{verbatim}
Created:    6 April 2023 at 17:07
Modified:   29 May 2023 at 11:39
Status: In Review
Voortgang:  1st draft
\end{verbatim}

What do we deem necessary in order to maintain faithfulness to reality
when exchanging data, and how can we assure that the data exchange does
not introduce \emph{phantom semantics}? In this reading, Maxim 2
(Quality) states that the DSP must assure to only exchange data that are
faithful to reality, i.e., reflect the current SoA properly; on her
turn, the DSC must assure to comprehend the exchanged data into semantic
meaning in a way that reflects the intended SoA. This is about semantic
accuracy, semantic validity, and being informed in good time. We assume
software agents to not break this maxim on purpose, but that does not
mean that the conversation does not introduce quality issues. Quality
issues can emerge from the so-called false-agreement problem (FAP)
\citep[Sec.2.4]{Guarino:1998wq}, resulting in collaborating agents to
falsely belief that they agree.

\begin{mmconcern}[Semantic faithfulness concern]\label{cncrn:qlymc}
When two or more actors communicate, phantom semantics, and notably the false-agreement problem can easily emerge. It is essential that the semantic meaning that is comprehended by the DSC matches the semantic meaning that is intended by the DSP.
\end{mmconcern}

Applied DP:

\begin{itemize}
\tightlist
\item
  DP3: Minimise FAP principle
\end{itemize}

\subsubsection{Conversation purpose}\label{conversation-purpose}

\begin{verbatim}
Created:    6 April 2023 at 17:07
Modified:   7 June 2023 at 16:03
Status: In Review
Voortgang:  1st draft
\end{verbatim}

In our sIOP interpretation of Maxim 3 (Relation), it demands that agents
need to communicate purposefully, relevant to the actual situation.

We distinguish at least different levels in agent comprehension:

\begin{itemize}
\tightlist
\item
  Comprehension about the communication protocols in order to engage in
  data exchange;
\item
  Comprehension about the semantic protocols in order to engage in
  exchanging semantic meaning;
\item
  Comprehension about the business collaboration in order to enable
  agent coordination towards the shared objective.
\end{itemize}

Apart from these levels in comprehension, two main distinctions in the
domain of discourse exist:

\begin{itemize}
\tightlist
\item
  Discourse in terms of the core of the domain, viz the SoAs that apply
  in the DoA;
\item
  Discourse \emph{about} the DoA, viz the meta-domain:

  \begin{itemize}
  \tightlist
  \item
    provenance of the domain things, e.g., when and where did it
    originate, why was it registered (privacy regulations such as GDPR,
    CCPA) and when and by whom was it modified / addressed;
  \item
    reliability of the statements made about the domain;
  \item
    authorisation about the SoA that apply, i.e., who what authorised to
    access what SoA, by whom and on the bases of what criteria;
  \item
    administration involved with the domain, e.g., required certificates
    or regulatory demands;
  \item
    and many more
  \end{itemize}
\end{itemize}

From an agent communication perspective, these meta-domains can be
considered domains of their own since the discourse will address a
particular meta-domain purpose to achieve. In such discourse, SoAs that
apply in the core domain are simple facts that are plugged as object
clause into the meta-domain discourse. These meta-domains require
vocabularies of their own which can be defined, or have been already;
PROV-O being an perfect example. This requires agents to ask or answer
questions about the purpose of the interaction with any other agent.

\begin{mmconcern}[Conversation purpose concern]\label{cncrn:rmc}
An agent needs to have the capability to engage in a meta-conversation -with one or more collaborating agents- with the single purpose to establish agreement on what to achieve with the conversation that they engage in. This capability can be considered a very fundamental capability when we consider the many different situations a group of agents can encounter, each requiring their own capabilities.
\end{mmconcern}

Applied DPs:

\begin{itemize}
\tightlist
\item
  DP X: Pragmatic protocol principle
\end{itemize}

So what? Based on this capability, any agent can engage in a
conversation -on behalf of a stakeholder- with any other agent and:

\begin{itemize}
\tightlist
\item
  requests to engage in a conversation to establish a particular
  purpose;
\item
  establish what it takes to establish that purpose
\item
  achieve that purpose, providing the conditions are satisfied
\end{itemize}

\subsubsection{Semantic representation}\label{semantic-representation}

\begin{verbatim}
Created:    6 April 2023 at 17:07
Modified:   29 May 2023 at 13:53
Status: In Review
Voortgang:  1st draft
\end{verbatim}

How can we understand each other('s data)?

\begin{itemize}
\tightlist
\item
  Data are representations, and carry semantic meaning implicitly.
\item
  Without (agreement on) a universal representation on semantic meaning,
  i.e., semantic homogeneity, agreements on the relation between each
  other's semantic representations must be established at runtime.
\item
  Since a representation is subsequent to a conceptualisation, an
  agreement on semantic representation implies compatibility between
  conceptualisations, viz similar OCs
\end{itemize}

\begin{mmconcern}[Semantic representation concern]\label{cncrn:mmc}
Collaborating agents in a semantic heterogeneous environment need to possess the capability to achieve a mutual understanding of how the SoAs of the DoA are being represented by their data.  
\end{mmconcern}

We consider the overall objective of sIOP as to exchange semantic
meaning about the DoA in order to progress towards the shared objective
of the collaboration. This represents one of the purposes that can be
vocalised by DP X, being the semantic purpose, viz to establish sIOP.
Such requires a semantic universe of discourse of its own. In order to
exchange semantic meaning, first, agreement must be reached whether the
required level of semantic compatibility can be achieved, based on what
OC and how. Second, agents must establish agreements on the vocabularies
and ontologies in use and the necessary alignments between them, or, in
absence of the latter, agents must engage in a reconciliation process
\citep{Diggelen:2007vd}. Only then both agents are prepared to engage in
exchanging semantic meaning.

Applied DPs:

\begin{itemize}
\tightlist
\item
  DP2: Semantic responsibility principle
\item
  DP4: Semantic protocol principle
\end{itemize}

So what? Based on this capability, and from a semantic perspective,
agents can achieve interoperability at the semantic level, viz making
preparations to \emph{engage} in exchanging semantic meaning, and to
actively contribute to \emph{consolidate} such capability, subsequently.
In terms of DP X, the pragmatic protocol principle, this means that any
agent can engage in a conversation -on behalf of a stakeholder- with any
other agent and:

\begin{itemize}
\tightlist
\item
  requests to talk about the semantic level
\item
  establish what it takes to establish sIOP
\item
  establish sIOP, providing the conditions are satisfied
\end{itemize}

\section{Reconciliation concerns}\label{reconciliation-concerns}

\begin{verbatim}
Created:    6 April 2023 at 17:07
Modified:   29 June 2023 at 15:47
Status: In Progress
Voortgang:  1st draft
\end{verbatim}

Semantic interoperability is about at least two software agents, in
their roles as DSP and DSC, that share a particular DoA and exchange
data that refer to a certain SoA in their shared reality. Since semantic
heterogeneity is assumed, achieving semantic reconciliation becomes a
necessary condition for sIOP.

\begin{mmdef}[Semantic reconciliation]\label{def:semantic-reconciliation}
Semantic reconciliation aligns (establishing correspondence and consistency) and harmonizes (resolving conflicts and achieving unity) differences in software agents' conceptualizations and representations of a shared domain. Its goal is to foster effective communication by establishing semantic homogeneity among agents' diverse ontological models. Semantic reconciliation relies on genuine understanding, which necessitates human involvement.
\end{mmdef}

The architectural concerns revolve around the tension that exists
between semantic reconciliation with its required human involvement and
its inevitable impact on the ``access-and-play'' business demand, i.e.,
minimising human effort to the highest extent possible.

\subsection{Semantic coherence}\label{semantic-coherence}

\begin{verbatim}
Created:    6 April 2023 at 17:07
Modified:   29 June 2023 at 21:18
Status: In Progress
Voortgang:  1st draft
\end{verbatim}

In \cref{semantics-in-software}, we defend the central disposition of
reciprocity between data and data processing, emerging as an Atomic
Semantic Monolith (ASM), with the explicit purpose to guarantee the
coherence between the data (semantic meaning) and their processing code
(pragmatic meaning).

Data exchange implies the DSP to tear down the ASM, separating the
semantic meaning from its pragmatic meaning. Since data comprehension
precedes its faithful use, the DSC must establish a valid reciprocity
between the DSP's semantic meaning with its own pragmatic meaning, e.g.,
when exchanging temperature, assure that the DSP's semantic meaning and
the DSC's pragmatic meaning apply similar background knowledge about
scales and units and alike, and assure that the temperature applies to
the same object, to name a few. Data exchange, thus, implies the DSC to
receive semantic meaning only, losing every guarantee that reciprocity
between the DSP's data and the DSC's data processing is coherent. Unless
it can be guaranteed that this reciprocity is as coherent as necessary
for a faithful comprehension by the DSC, it is impossible to establish
sIOP without emergence of phantom semantics. We consider maintaining
semantic coherence a necessary condition for sIOP to emerge, and
formulate the third concern about sIOP as follows.

\begin{mmconcern}[Semantic coherence concern]\label{cncrn:re-establish-coherence}
The reciprocity between data and their processing is the carrier of RWS. By the nature of data exchange, this reciprocity is broken for the DSC on receipt of data. Without reciprocity, no faithful comprehension of the DSP’s semantic meaning can occur; absence of faithful comprehension by the DSC impedes its capability to establish sIOP with the DSP.
\end{mmconcern}

Applied DPs:

\begin{itemize}
\tightlist
\item
  DP8: Alignment principle.\\
\item
  DP9: Correspondence relation principle
\end{itemize}

\subsection{Human involvement}\label{human-involvement}

\begin{verbatim}
Created:    6 April 2023 at 17:07
Modified:   29 June 2023 at 21:18
Status: In Progress
Voortgang:  1st draft
\end{verbatim}

The purpose of comprehension is to reconcile semantic differences, which
remains a distinct concern. Its resolution will depend on the position
that is taken on comprehension, and from our position its resolution
requires human effort to understand and resolve the semantic differences
between the interoperating agents. The architectural issue then becomes
where to position the dependency on the human when considering
\cref{cncrn:re-establish-coherence}, in order to minimise her effort.

\begin{mmconcern}[Human involvement concern]\label{cncrn:hic}
Provided that the human capability to genuinely understand semantics is required to reconcile semantic differences in a heterogeneous semantic environment, what will be her responsibilities, how can the dependency on her become minimal, and what tools should be available to her support?
\end{mmconcern}

Applied DP:

\begin{itemize}
\tightlist
\item
  DP10: Human efficiency principle
\end{itemize}

\subsection{Business coordination}\label{business-coordination}

\begin{verbatim}
Created:    6 April 2023 at 17:07
Modified:   25 May 2023 at 17:44
Status: In Progress
Voortgang:  1st draft
\end{verbatim}

The objective of collaborating stakeholders is to establish, in concert,
the projected SoA that represents the goal of the business
collaboration. Due to its collaborative nature, each different
stakeholder contributes to the projected SoA. Such an individual
contribution is produced by agent's individual telic behaviour, and is
represented as an individual SoA. This brings about the following
concern:

\begin{mmconcern}[Business coordination concern]\label{cncrn:bcc}
Due to its collaborative nature, the goal of business collaboration, viz the shared projected SoA, depends on the individual SoAs of each stakeholder, which on their turn might be dependent, physically and/or temporally, on other agents’ individual SoAs. Provided that an individual SoA represents the result of a particular telic behaviour of a specific agent, what semantic meaning do the contributing agents need to exchange in order to coordinate these individual telic behaviours in a dynamical business environment?
\end{mmconcern}

Applied DP:

\begin{itemize}
\tightlist
\item
  DP11: Business conversation protocol principle
\end{itemize}

Collaborating towards shared business objectives occur between agents,
and their communication will proceed through different stages.
Sometimes, these stages will follow a determined sequence, sometimes a
stage change will emerge in an ad-hoc fashion to account for the
occurrence of unexpected states of affairs (SoAs). In order to support a
dynamic collaboration, it is necessary for agents to gain awareness of
the purpose of the conversation, to consolidate an effective
conversation and adapt their conversational contribution when necessary.

These situations can differ hugely, when \emph{engaging in} business
collaboration or \emph{adapting to} changes. Different stages of
collaboration require different conversations with different
conversation objectives, i.e., to recognise the stage of collaboration,
to determine the objective of the cooperation at that stage, to know
about the conversation that such stage will ensue and about the criteria
for a transition to another stage.

\section{Mediation concerns}\label{mediation-concerns}

\begin{verbatim}
Created:    6 April 2023 at 17:07
Modified:   25 May 2023 at 20:46
Status: In Progress
Voortgang:  1st draft
\end{verbatim}

In the previous sections we identified the need for agents to exchange
semantic meaning in their local representation of their own
conceptualisation that only applies their locally adopted background
knowledge. We identified several semantic reconciliation concerns and
introduced alignments to their resolution, established how to minimise
human involvement, and provided a mechanism to coordinate each agent's
telic behaviour into collaboration. These are all preparations in order
to achieve the actual collaboration between agents, and it is in this
reading that we again apply the cooperative principle from
\cref{cooperative-concern}. This results in two distinct capabilities
that

However, we als need to apply Maxim 3, relation, because each partners'
contribution must be appropriate to the immediate needs at each stage of
the conversation. We define the agents' cooperation in terms of the
interaction between them, as follows.

\begin{mmdef}[Interaction coordination]\label{def:ic}
Interaction coordination is the process of directing the particular flows of conversation that occur in a collaboration, specifically about which agents need to cooperate by what semantic meaning in what order, initiating telic behaviour from the concerned agents, required to achieve the mutual objective of the collaboration.
\end{mmdef}

Semantic mediation can now be defined as:

\begin{mmdef}[Semantic mediation]\label{def:sm}
Semantic mediation is a generic, run-time capability that transcribes data between the native DMs of the collaborating agents, and that directs the particular flows of conversation that occur in a collaboration.
\end{mmdef}

\subsection{Data transcription}\label{data-transcription}

\begin{verbatim}
Created:    20 April 2023 at 12:15
Modified:   25 May 2023 at 20:46
Status: In Progress
Voortgang:  Writing
\end{verbatim}

In response to Maxim 1, the quantity maxim (``Make your contributions as
informative as is required (for the current purpose of the exchange),
and not more than is required''), transcription is a key process in
establishing sIOP in heterogeneous environments. To assure that the DSP
and DSC communicate as informative as is required, both must produce and
consume data that comply to their own native conceptualisation and
representation. Indeed, the DSP discloses with its semantic meaning only
what it deems relevant and exactly as it fits in its own DM,
\cref{quantity-maxim}. Receiving this from the DSP agent, it is quite
incomprehensible for any DSC agent. To make the DSP's contribution as
informative as is required for the DSC, a transcription of semantic
meaning is required.

Furthermore, Maxim 4, the manner maxim, demands that the data exchange
must avoid obscurity of expression, ambiguity. We introduce
\emph{phantom semantics} as transcription-induced statements (data) that
the DSC agent will interpret to reflect state of affairs that are not
intended by the DSP agent. Then, occurrence of phantom semantics should
be considered a communication failure.

Combining both maxims in this way, the following concern ensues:

\begin{mmconcern}[Data transcription concern]\label{cncrn:dtc}
    sIOP requires to transcribe the exchanged statements (data) between the native DMs of collaborating agents without introducing phantom semantics. 
\end{mmconcern}

Such transcription capability provides the operational capability to
mediate all differences that occur in the conceptualisation and
representation of semantic meaning that is carried by the data being
exchanged between the DSP and the DSC.

Applied DP:

\begin{itemize}
\tightlist
\item
  DP12: Data transcription principle
\end{itemize}

\subsection{Interaction coordination}\label{interaction-coordination}

\begin{verbatim}
Created:    6 April 2023 at 17:07
Modified:   25 May 2023 at 20:46
Status: In Progress
Voortgang:  Writing
\end{verbatim}

A business collaboration between agents can become quite complex in
terms of interactions that are required to achieve the shared business
objective. It surfaces the equal demand from each agent to manage and
control a multitude of semantic interactions in order to play its own
part in all collaborations that it is engaged in.

Individual agents don't have a need to be intimately aware of each
other; it is sufficient for the agents to deliver what is requested and
to receive what they require. Controlling and coordinating the data
streams depends on the conversation that applies, and although that
conversation must be coordinated and managed, there is no reason for
agents to take upon them that responsibility. They \emph{do} have
responsibility on behalf of their own, local pragmatic meaning; when
they are in need of additional SoA's in order to comprehend received
data, to finalise a conversation or to achieve their own objective, they
have the responsibility to address that demand, using either local or
remote knowledge.

\begin{mmconcern}[Interaction coordination concern]\label{cncrn:icc}
To manage this complexity, it must be resolved in a way that administers justice to the shared and the individual responsibilities and to support to the changes that will occur in a dynamic business environment. The concern refers to coordinating the flow of interactions, i.e., data exchange, between collaborating agents that occur during their business conversations. 
\end{mmconcern}

We consider this concern related to but different from
\cref{cncrn:scalability}, as it surfaces the complexity of the business
collaboration process as another scalability dimension.

The issue at hand is to untwine the responsibilities between those that
can and can not be delegated to other agents or infrastructural service
providers. And although its primary focus is on responsibilities in
relation to business collaboration and their conversations, it is also
relevant for responsibilities that come into play during the entrance of
new agents and exit of existing agents. Clearly, the latter is driven by
the scalability concern, and we will address those responsibilities in
\cref{concern-sc1-emantic-scalability-concern}.

Applied DP:

\begin{itemize}
\tightlist
\item
  DP13:Interaction coordination principle
\end{itemize}

\section{Scalability concerns}\label{scalability-concerns}

\begin{verbatim}
Created:    6 April 2023 at 17:07
Modified:   25 May 2023 at 17:24
Status: In Progress
Voortgang:  Writing
\end{verbatim}

Although they share semantic meaning, the resulting semantic monolith of
the DSC must be allowed to differ from the DSP's semantic monolith. In
fact, collaboration is all about sharing data that carry similar
semantic meaning but allowing for different pragmatic meaning in order
to achieve different results that together fulfils the shared purpose.
For example, by exchanging a heartbeat both agents share the semantic
meaning about the number of beats per second. However, the pragmatic
meaning can vary between an indication of health for a health-care
application and an indication of performance potential in a sports
application, with different pragmatic demands on semantic meaning, e.g.,
resolution or accuracy. Additionally, and as identified as
\cref{cncrn:semantic-heterogeneity}, semantic meaning can be represented
in many different ways. Both aspects bring about a further concern on
semantic scalability, since semantics won't be centrally coordinated
anymore resulting in semantic definitions that are distributed all over
the place (see \cref{dp:shp}).

In our pursuit to formulate this scalability concern, we first provide
for a definition that draws from the classical formulations on system
scalability \citep{Bondi2000, Steed2010, Shivakumar2015}. In its
classical interpretation, scalability is about increasing quantity while
maintaining the quality of the system, in terms of performance,
functionality or efficiency as perceived by the end users. In the same
spirit, we address semantic scalability also in terms of quantity, where
the number of perspectives on reality have the most impact on semantic
diversity and representational variety, more than stakeholders alone.
Furthermore, we formulate the quality to maintain not in terms of
interoperability itself, but the effort that is required to consolidate
sIOP between those many perspectives, notably since this addresses the
main feature that sIOP is to support: business agility. This results in
the following definition:

\begin{mmdef}[Semantic scalability]\label{def:semantic-scalability}
Semantic scalability is the capability of a system of collaborating software agents to adopt and/or consolidate, with acceptable lead time and costs, additional perspectives on their shared reality without compromising sIOP between the interacting software agents. 
\end{mmdef}

\subsection{Semantic scalability}\label{semantic-scalability}

\begin{verbatim}
Created:    6 April 2023 at 18:09
Modified:   25 May 2023 at 20:45
Status: In Progress
Voortgang:  Writing
\end{verbatim}

Pursuing business agility requires semantic scalability; pursuing
business agility in an open world results in heterogeneity and, hence,
requires to cope with semantic independence. We formulate the following
concern to its effect:

\begin{mmconcern}[Semantic scalability]\label{cncrn:scalability}
  In an open world, semantic definitions are distributed and commit to different perspectives on the DoA while central coordination is not always available. Furthermore, computers lack the capability for genuine understanding. Consequently, semantic differences between collaborating agents must be explicitly reconciled and maintained, implying semantic agreements in support of sIOP. Agreements imply a relationship to exist between software agents, as well as institutionalising (the governance of) those relationships to manage them under continuing changes in business collaborations and the occurrence of semantic drift. At the same time, semantic scalability requires collaborative software agents to maximise their semantic independence.
\end{mmconcern}

Applied DP:

\begin{itemize}
\tightlist
\item
  DP14: Semantic scalability principle
\end{itemize}

\subsection{Loosely coupled semantic}\label{loosely-coupled-semantic}

\begin{verbatim}
Created:    6 April 2023 at 17:07
Modified:   25 May 2023 at 17:24
Status: In Progress
Voortgang:  Writing
\end{verbatim}

Finally, we consider the classical concern about loose coupling but in a
semantic reading instead, in order to respond to consequences that
follow from pursuing interoperability while insisting on a semantic
heterogeneous environment.

Finally, we consider the classical concern about loose coupling but in a
semantic reading instead, in order to respond to consequences that
follow from pursuing interoperability while insisting on a semantic
heterogeneous environment. In software, how does data relate to
real-world semantics?

DEFINITIE VAN LCS TOEVOEGEN

By introducing semantic heterogeneity, we renounce to bind domain
semantics to particular syntax upfront. Nevertheless, stakeholders must
remain the capability to use exchanged data, hence, semantic coupling
must be realised. Loose coupling is known as a strong architectural
principle that brings about many advantages. This holds for its semantic
reading as well: agents that are loosely coupled in their semantics (i)
can communicate in their own native representations without the need to
learn or integrate foreign representations; (ii) need to define
semantics only once for a particular DoA, and use it in concert with
every software agent that operates in that DoA; (iii) can define their
semantic representations geared to fit their particular domain and
application(s) in isolation from global semantics; and (iv) can re-use
available, local semantics, improving the economic value of local
semantics throughout their lifetime. Moreover, since re-use implies
another confrontation with reality, (v) the quality and scope of the
local semantics will improve and extend over time.

In its classical sense, loose coupling is realised through the
principles of separation of concerns and transparency. We investigate
how to apply those principles at the semantic level in the next two
subsections.

\subsubsection{Semantic transparency}\label{semantic-transparency}

\begin{verbatim}
Created:    6 April 2023 at 17:07
Modified:   25 May 2023 at 20:45
Status: In Progress
Voortgang:  Writing
\end{verbatim}

Definitie van semantic transparency toevoegen; het concern is daarvan
afgeleid: ``What is required to enforce semantic transparency?''

\begin{mmconcern}[Semantic transparency concern]\label{cncrn:stc}
TBD
\end{mmconcern}

Applied DP:

\begin{itemize}
\tightlist
\item
  DP7: Semantic transparent API principle
\end{itemize}

\subsubsection{Semantic separation of
concerns}\label{semantic-separation-of-concerns}

\begin{verbatim}
Created:    6 April 2023 at 17:07
Modified:   25 May 2023 at 20:45
Status: In Progress
Voortgang:  Writing
\end{verbatim}

\begin{mmconcern}[Semantic separation of concerns]\label{cncrn:ssoc}
TBD
\end{mmconcern}

Applied DPs:

\begin{itemize}
\tightlist
\item
  DP4: Semantic protocol principle
\item
  DP5: Separate data syntax from semantics syntax principle
\item
  DP6: Semantic modularity principle
\end{itemize}

\section{Operationalisation}\label{operationalisation}

\begin{verbatim}
Created:    1 June 2023 at 13:41
Modified:   1 June 2023 at 14:46
Status: In Review
Voortgang:  1st draft
\end{verbatim}

To be done by Eric.

\emph{PREPARATION}

Apply an architectural method, e.g., ArchiMate, to direct the
operationalisation. From that method, identify those elements from the
method (activities, phases, concepts, whatever) that apply to sIOP. For
each of those elements, identify the DP that is relevant to it. This
results in a matrix of relevant architectural elements versus DPs. Check
that all DPs are categorised to at least one element; if one is missing,
either the matrix is incomplete or there is something missing in the
architectural method. Correct the matrix and/or set the unused DPs
apart.

\emph{WRITING}

\begin{itemize}
\tightlist
\item
  The structure of this main section follows from the elements of
  interest: each relevant element justifies a single section, and - if
  necessary - sections can be structured according to the containing
  phase or whatever from the method.
\item
  Introduce this main section by describing:

  \begin{itemize}
  \tightlist
  \item
    Purpose of the section, i.e., showing that it is possible to
    operationalise the DPs and that the agents' sIOP capabilities will
    be extended;
  \item
    Method that is followed, i.e., selection of a particular view-based
    architectural method, why this particular method is chosen, and that
    we integrate the DPs in it;
  \item
    Present the matrix as overview of how the main section is broken
    down in its sections/subsections;
  \end{itemize}
\item
  In each section, answer:

  \begin{itemize}
  \tightlist
  \item
    Purpose: What is the purpose of this element in the architectural
    method?
  \item
    Rationale: Why is it relevant for this element to address this
    DP(s)?
  \item
    Application: What steps are involved in applying the DP(s)? The
    method should direct the steps as the baseline in which the DPs are
    to be merged. Focus on what the step achieves as opposed to how it
    is done or the technology involved.
  \item
    Consequence: What sIOP capability is gained for the software agent
    that was not possible before?
  \end{itemize}
\item
  Finalise with a section - if necessary - that explains how the
  architectural method can be extended ot include the unaddressed DPs.
\end{itemize}

\section{DPs}\label{dps}

\begin{verbatim}
Created:    6 April 2023 at 17:02
Modified:   6 April 2023 at 17:33
Status: No Status
Voortgang:  No Label
\end{verbatim}

\subsection{DP1:shp Semantic Heterogeneity
Principle}\label{dp1shp-semantic-heterogeneity-principle}

\begin{verbatim}
Created:    6 April 2023 at 17:07
Modified:   23 April 2023 at 23:13
Status: In Review
Voortgang:  Revised draft
\end{verbatim}

sIOP should strive to support multiple co-existing perspectives of use
as opposed to enforcing one single perspective, and should be founded on
the active reconciliation of semantic differences rather than on
allowing one homogeneous semantic standard only.

\subsection{DP: Separate DM from SM}\label{dp-separate-dm-from-sm}

\begin{verbatim}
Created:    6 April 2023 at 17:07
Modified:   23 April 2023 at 23:10
Status: No Status
Voortgang:  1st draft
\end{verbatim}

Gedefinieerd in paper 1 als een combinatie van:

\begin{itemize}
\tightlist
\item
  Semantic Coherence Principle: Establish explicit coherence between the
  data (S-model) and the data-processing (P-model).
\item
  Semantic Grounding Principle: All models that are part of the System
  Model and refer to the DoA, are grounded in the Domain Model.
\end{itemize}

\subsection{DP: Separate comprehension from telic
behaviour}\label{dp-separate-comprehension-from-telic-behaviour}

\begin{verbatim}
Created:    6 April 2023 at 17:07
Modified:   23 April 2023 at 23:11
Status: In Review
Voortgang:  1st draft
\end{verbatim}

Defined in paper 1 als: Make a distinction between comprehension
operations that follow from the inferential model to establish pragmatic
meaning (data-processing for comprehension) and telic operations that
realise actions in order to achieve a certain goal of the application
(data-processing to produce results).

\subsection{DP2:srp Semantic responsibility
principle}\label{dp2srp-semantic-responsibility-principle}

\begin{verbatim}
Created:    6 April 2023 at 17:07
Modified:   23 April 2023 at 23:13
Status: In Review
Voortgang:  Revised draft
\end{verbatim}

The software agent has the responsibility to explicate and disclose its
OC, and, additionally, to disclose the semantic meaning of any exchanged
data, to the extent necessary to allow comprehension of its data by the
other involved agent(s).

\subsection{DP3:mfapp Minimise FAP
principle}\label{dp3mfapp-minimise-fap-principle}

\begin{verbatim}
Created:    6 April 2023 at 17:07
Modified:   23 April 2023 at 23:13
Status: In Review
Voortgang:  Revised draft
\end{verbatim}

In order to minimise the False-Agreement Problem, the software agent's
domain model must be constrained to minimize the difference between the
semantic meaning that it allows and those that it intends.

\subsection{DP4:spp Semantic protocol
principle}\label{dp4spp-semantic-protocol-principle}

\begin{verbatim}
Created:    6 April 2023 at 17:07
Modified:   5 May 2023 at 15:10
Status: In Review
Voortgang:  Revised draft
\end{verbatim}

The agent should provide the capability to respond to a semantic
protocol that has the objective to prepare sIOP. First, agents must
reach agreement whether the required level of semantic compatibility can
be achieved, and how. Second, agents must establish agreements on the
{[}vocabularies and ontologies \textbar{} domain models{]} in use.
Third, agreement must be reached about how they relate semantically to
each other; this might include engaging in a reconciliation process (van
Diggelen, 2007), and/or application of pre-determined alignments between
them.

\subsection{DP5:sds-ss Separate data syntax from semantics syntax
principle}\label{dp5sds-ss-separate-data-syntax-from-semantics-syntax-principle}

\begin{verbatim}
Created:    6 April 2023 at 17:07
Modified:   23 April 2023 at 23:14
Status: In Review
Voortgang:  Revised draft
\end{verbatim}

Separate syntax representing the data payload in the agent
interoperation from syntax representing the semantic payload of the
data.

\subsection{DP6:smp Semantic modularity
principle}\label{dp6smp-semantic-modularity-principle}

\begin{verbatim}
Created:    6 April 2023 at 17:07
Modified:   23 April 2023 at 23:14
Status: In Review
Voortgang:  Revised draft
\end{verbatim}

When it is reasonable to expect that the domain model (DM) of the agent
will grow throughout its lifetime, the DM should allow for
modularisation methods that simplify and downsize the DM into manageable
parts.

\subsection{DP7:stp Semantic transparent API
principle}\label{dp7stp-semantic-transparent-api-principle}

\begin{verbatim}
Created:    6 April 2023 at 17:07
Modified:   23 April 2023 at 23:14
Status: In Review
Voortgang:  Revised draft
\end{verbatim}

The software agent must remain agnostic to the local
conceptualisation(s) and representation(s) of the agent(s) it
collaborates with by implementing APIs that access semantic
functionalities without committing to the particular domain semantics.
Standardise communication services and their parameters on the semantic
metalevel only, and only allow domain semantics as payload to the API.

\subsection{DP8:ap Alignment principle}\label{dp8ap-alignment-principle}

\begin{verbatim}
Created:    6 April 2023 at 17:07
Modified:   23 April 2023 at 23:14
Status: In Review
Voortgang:  Revised draft
\end{verbatim}

Align semantic meanings, not data schemata or data syntax.

\subsection{DP9:crp Correspondence relation
principle}\label{dp9crp-correspondence-relation-principle}

\begin{verbatim}
Created:    6 April 2023 at 17:07
Modified:   23 April 2023 at 23:14
Status: In Review
Voortgang:  Revised draft
\end{verbatim}

A correspondence relation must preserve semantic meaning by addressing
conceptual differences as well as convention-induced value-space
differences. The alignment language must be expressive enough to
represent those differences, the resolution of which is a necessary
condition to achieve sIOP, providing faithful DMs from both
collaborating agents.

\subsection{DP10:hep Human efficiency
principle}\label{dp10hep-human-efficiency-principle}

\begin{verbatim}
Created:    6 April 2023 at 17:07
Modified:   23 April 2023 at 23:14
Status: In Review
Voortgang:  Revised draft
\end{verbatim}

The human responsibility in semantic reconciliation should, to the
extent possible, focus on auditing semantic alignments, and minimize
their authoring. Maximise the use of reconciliation and auditing tools
as well as the reuse of available alignment information.

\subsection{DP11:bcpp Business conversation protocol
principle}\label{dp11bcpp-business-conversation-protocol-principle}

\begin{verbatim}
Created:    6 April 2023 at 17:07
Modified:   23 April 2023 at 23:14
Status: In Review
Voortgang:  Revised draft
\end{verbatim}

The agent should provide the capability to engage in a business
conversation with participating agents, with the objective to jointly
achieve a conversation state that is considered a(n intermediate,)
necessary step towards the collaboration goal.

\subsection{DP12:dtp Data transcription
principle}\label{dp12dtp-data-transcription-principle}

\begin{verbatim}
Created:    6 April 2023 at 17:07
Modified:   27 April 2023 at 12:45
Status: No Status
Voortgang:  1st draft
\end{verbatim}

Transcribe exchanged data between their local and remote
conceptualisation and representation, without introducing phantom
semantics. Apply for its implementation the mediation pattern to
instantiate a business conversation based transcription component that
is configured by semantic alignments.

\subsection{DP13:icp Interaction coordination
principle}\label{dp13icp-interaction-coordination-principle}

\begin{verbatim}
Created:    6 April 2023 at 17:07
Modified:   27 April 2023 at 12:37
Status: No Status
Voortgang:  1st draft
\end{verbatim}

The coordination and control of business interactions that apply in the
context of a particular business conversation. must either be
encapsulated by the software agent or assumed as an infrastructural
mediated service, preferring the latter. Both solutions should
instantiate a an interaction coordination session at the start of a
business conversation, configured either by a business choreography, a
business orchestration or a business conversation specification. The
session ends at achieving the projected SoA. The session is to maintain
a cache of the SoAs that apply as they occur or change. Semantics and
Interaction are assumed to remain stable over a single session, but can
change between sessions.

\subsection{DP14:ssp Semantic scalability
principle}\label{dp14ssp-semantic-scalability-principle}

\begin{verbatim}
Created:    6 April 2023 at 17:07
Modified:   23 April 2023 at 23:17
Status: To Do
Voortgang:  Writing
\end{verbatim}

TBD

%%%%%%%%%%%%% APPENDICES
%% !ToDo!: Implement appendices.
%%
%% The Appendices part is started with the command \appendix;
%% appendix sections are then done as normal sections, i.e.,
%% \appendix
%% \section{}
%% \label{}

%%%%%%%%%%%%% ACKNOWLEDGMENTS
\section*{Acknowledgments}
``\,''


%%%%%%%%%%%%% BIBLIOGRAPHY
%% If you have bibdatabase file and want bibtex to generate the
%% bibitems:
%% 1. Set the `bibliography` parameter in the pandocomatic configuration;
%% 2. Respect the related remarks from pandocomatic-elsarticle.template.
%%
\bibliography{/Users/paulbrandt/Documents/Literature/CitedByMe-2021-archSIOp.bib}
%%


\end{document}

%%
%% Note 1: For some reason that I cannot diagnose, the package `stackengine`
%%         is required with the `\longtable` command as it is being used in 
%%         elsarticle.latex; without its inclusion, the following error is
%%         thrown, referring to the first four lines after `\longtable`:
%% ! Undefined control sequence.
%% <argument> (\columnwidth - 6\tabcolsep ) * \real 
%%                                                  {0.2893}
%%         Anyone who can diagnose the source of this error, please inform me.
%%
%% End of file `elsarticle.latex'.
